\documentclass[12pt,leqno,letterpaper]{article}


% PAGE SETTINGS %

\parskip 0in
\parindent 0.32in
%\voffset -0.125in

\usepackage[letterpaper]{geometry}
\geometry{text={5.562306in,9in}, centering} % golden ratio :)
%\renewcommand{\baselinestretch}{1.02} 

% FONTS %

\usepackage{ifxetex}
\ifxetex
  \usepackage[T1]{fontenc}
  \usepackage[no-math]{fontspec}

  \setmainfont[Scale=1.0]{Times NR MT Std}

  \usepackage[cal=euler, calscaled=1,
	      scr=boondox, scrscaled=1]{mathalfa}

  \usepackage[mtpcal,mtphrb,mtpfrak,noamssymbols,nofontinfo,
  	      subscriptcorrection,zswash]{mtpro2}

  %\DeclareMathSizes{12}{13}{9}{7}
\else
  \usepackage{lmodern}
  \usepackage{amsfonts}
  \usepackage{amssymb}
  \usepackage{mathrsfs}
  % to match mt2pro
  \newcommand{\mbf}{\mathbf} 
  \newcommand{\what}[1]{\widehat{#1}}
\fi

% THEOREMS %

\usepackage{amsthm}

\newtheoremstyle{mytheoremstyle}
{\topsep}                    % Space above
{\topsep}                    % Space below
{\itshape}                   % Body font
{}                           % Indent amount
{\bfseries}                  % Theorem head font
{.}                          % Punctuation after theorem head
{.5em}                       % Space after theorem head
{}  % Theorem head spec (can be left empty, meaning ‘normal’)

\theoremstyle{mytheoremstyle}

\makeatletter
\@ifclassloaded{report}{
   \newtheorem{theorem}{Theorem} }
{  \newtheorem{theorem}{Theorem}[section] }
\makeatother

\newtheorem{thm}[theorem]{Theorem}
\newtheorem{proposition}[theorem]{Proposition}
\newtheorem{lemma}[theorem]{Lemma}
\newtheorem{corollary}[theorem]{Corollary}
\newtheorem{definition}[theorem]{Definition}
\newtheorem{remark}[theorem]{Remark}
\newtheorem{example}[theorem]{Example}
\newtheorem{condition}[theorem]{Condition}
\newtheorem{assumption}[theorem]{Assumption}

\newtheorem*{theo}{Theorem}
\newtheorem*{lemm}{Lemma}
\newtheorem*{defn}{Definition}
\newtheorem*{prop}{Proposition}
\newtheorem*{assm}{Assumption}
\newtheorem*{cond}{Condition}
\newtheorem*{remk}{Remark}
\newtheorem*{expl}{Example}


% REFERENCES %

\usepackage{harvard}

% SECTIONS %

\usepackage{titlesec}

\titleformat{\section}[runin]
{\normalfont\bfseries}
{\thesection.}{0.08in}{}[.]

\titlespacing{\section}{\parindent}{0.08in}{0.08in}

\titleformat{\subsection}[runin]
{\normalfont\slshape}
{\thesubsection.}{0.04in}{}[.]

\titlespacing{\subsection}{\parindent}{0.08in}{0.08in}


% ABSTRACT

\renewcommand{\abstractname}{\vspace{-0.48in}}

\renewenvironment{abstract}
{\small
\begin{center}
  \bfseries \abstractname\vspace{-.5em}\vspace{0pt}
\end{center}
\list{}{
	\setlength{\leftmargin}{\parindent}
    \setlength{\rightmargin}{\leftmargin}
    \setlength{\parsep}{0in}
	\itemindent \parindent} 
  \item\relax}
{\endlist}

% TABLES AND FIGURES


\usepackage{caption}
\captionsetup[figure]{margin=0.32in,
	labelsep=period,labelfont=bf,textfont=it}
\captionsetup[table]{margin=0.32in,
	labelsep=period,labelfont=bf,textfont=it}





% BASIC PACKAGES %

\usepackage{makecell}
\usepackage{graphicx}
\usepackage{enumitem}
\usepackage{booktabs}
\usepackage{framed}
\usepackage{multirow}
\usepackage{scrextend}
\usepackage{stackengine}
\usepackage{subcaption}
\usepackage{comment}
\usepackage{url}
\usepackage{bm}
\usepackage[bottom]{footmisc}

% MARGIN NOTES %

\usepackage{marginnote}
\usepackage{setspace}

\newcommand{\lmn}[1]{{\reversemarginpar \setstretch{0.64} %
	\marginnote{{\scriptsize \bambootext{\emph{#1}}}}} }

\newcommand{\rmn}[1]{{\normalmarginpar \setstretch{0.64} %
	\marginnote{{\scriptsize \bambootext{\emph{#1}}}}} }

\setlength\marginparwidth{64pt}

% RAISE/LOWER %

\makeatletter
\newcommand{\raisemath}[1]{\mathpalette{\raisem@th{#1}}}
\newcommand{\raisem@th}[3]{\raisebox{#1}{$#2#3$}}
\makeatother

% HIDE MACRO %

\newif\ifhide
\hidetrue %\hidefalse
\newcommand{\hide}[1]{\ifhide {} \else {#1} \fi} 

% EMPHASIZE BOX %

\usepackage{empheq}
\newcommand{\widefbox}[1]{\fbox{\hspace{0.33in}#1\hspace{0.33in}}}


% CHAR MAPS: e.g. \calS for \mathcal{S}  %

\usepackage{pgffor}

\foreach \x in {A,...,Z}{%
  \expandafter\xdef\csname cal\x\endcsname{\noexpand %
	\ensuremath{\noexpand\mathcal{\x}}}
  \expandafter\xdef\csname scr\x\endcsname{\noexpand %
	\ensuremath{\noexpand\mathscr{\x}}}
  \expandafter\xdef\csname bb\x\endcsname{\noexpand %
	\ensuremath{\noexpand\mathbb{\x}}}
  \expandafter\xdef\csname rm\x\endcsname{\noexpand %
	\ensuremath{\noexpand\mathrm{\x}}}
  \expandafter\xdef\csname bf\x\endcsname{\noexpand %
	\ensuremath{\noexpand\mathbf{\x}}}
}

% ACCENTS %

\newcommand{\wh}[1]{\widehat{#1}}
\newcommand{\wt}[1]{\widetilde{#1}}

% SPACING %
\newcommand{\hs}[1]{ \hspace{#1pt} }

\newcommand{\mptt}{\hspace{-2pt}  }
\newcommand{\mpt}{ \hspace{-1pt}  }
\newcommand{\mhpt}{\hspace{-0.5pt}}
\newcommand{\hpt}{ \hspace{0.5pt} }

\newcommand{\pt}{ \hspace{1pt}  }
\newcommand{\ptt}{ \hspace{2pt} }
\newcommand{\pttt}{ \hspace{3pt}}
\newcommand{\ptttt}{\hspace{4pt}}

% BASIC MATH %

%\newcommand{\bst}{ \hspace{1.5pt} | \hspace{1.5pt} }
%\newcommand{\cst}{ \hspace{0.5pt} : \hspace{0.5pt} }
%\newcommand{\sst}{ \hspace{2pt} ; \hspace{0.5pt} }
\newcommand{\evl}{ \left| \right. }

\newcommand{\abs}[1]{\left| {#1} \right|}
\newcommand{\ceil}[1]{\left\lceil #1 \right\rceil}
\newcommand{\floor}[1]{\left\lfloor #1 \right\rfloor}

% ARROWS %

%\newcommand{\upto}{ \uparrow }
%\newcommand{\downto}{ \downarrow }
%\newcommand{\nto}{ \nrightarrow }

% TEXT %

%\newcommand{\tq}[1]{{\textquotedblleft #1\textquotedblright}}
%\newcommand{\ts}[2]{{#1\textsubscript{#2}}}


% Brackets, Parentheses, etc %

\newcommand{\pa}[1]{\hspace{1pt} \left( \hspace{0.5pt} %
	{#1} \hspace{0.5pt} \right) \hspace{1pt}}
\newcommand{\qa}[1]{\hspace{1pt} \left[ \hspace{0.5pt} %
	{#1} \hspace{0.5pt} \right] \hspace{1pt}}
\newcommand{\fa}[1]{\hspace{1pt} \left \{ \hspace{0.5pt} % 
	{#1} \hspace{0.5pt} \right \}\hspace{1pt}}
\newcommand{\za}[1]{\hspace{1pt} \left\langle \hspace{0.5pt} %
	{#1} \hspace{0.5pt} \right\rangle \hspace{1pt}}

\newcommand{\p}[1]{ \hspace{1pt} ( \hspace{0.25pt} %
	{#1} \hspace{0.25pt} ) \hspace{1pt}}
\newcommand{\q}[1]{ \hspace{1pt} [ \hspace{0.25pt} %
	{#1} \hspace{0.25pt} ] \hspace{1pt}}
\newcommand{\f}[1]{ \hspace{1pt} \{ \hspace{0.25pt} %
	{#1} \hspace{0.25pt} \} \hspace{1pt}}
\newcommand{\z}[1]{ \hspace{1pt} \langle \hspace{0.25pt} %
	{#1} \hspace{0.25pt}\rangle\hspace{1pt}}
\newcommand{\g}[1]{ \hspace{1pt} \langle \hspace{0.25pt} %
	{#1} \hspace{0.25pt}\rangle\hspace{1pt}}

\newcommand{\pp}[1]{ \hspace{1pt} \big( \hspace{0.5pt} %
	{#1} \hspace{0.5pt} \big) \hspace{1pt}}
\newcommand{\qq}[1]{ \hspace{1pt} \big[ \hspace{0.5pt} %
	{#1} \hspace{0.5pt} \big] \hspace{1pt}}
\newcommand{\ff}[1]{ \hspace{1pt} \big\{ \hspace{0.5pt} %
	{#1} \hspace{0.5pt} \big\} \hspace{1pt}}
\newcommand{\zz}[1]{ \hspace{1pt} \big\langle \hspace{0.5pt} %
	{#1} \hspace{0.5pt} \big\rangle \hspace{1pt}}

\newcommand{\ppp}[1]{ \hspace{1pt} \Big( \hspace{0.5pt} %
	{#1} \hspace{0.5pt} \Big) \hspace{1pt}}
\newcommand{\qqq}[1]{ \hspace{1pt} \Big[ \hspace{0.5pt} %
	{#1}  \hspace{0.5pt} \Big] \hspace{1pt}}
\newcommand{\fff}[1]{ \hspace{1pt} \Big\{ \hspace{0.5pt} %
	{#1}  \hspace{0.5pt} \Big\} \hspace{1pt}}
\newcommand{\zzz}[1]{ \hspace{1pt} \Big\langle \hspace{0.5pt} %
	{#1} \hspace{0.5pt} \Big\rangle \hspace{1pt}}

\newcommand{\pppp}[1]{ \hspace{1pt} \bigg(% 
	\hspace{0.5pt} {#1} \hspace{0.5pt} \bigg) \hspace{1pt}}
\newcommand{\qqqq}[1]{ \hspace{1pt} \bigg[% 
	\hspace{0.5pt} {#1} \hspace{0.5pt} \bigg] \hspace{1pt}}
\newcommand{\ffff}[1]{ \hspace{1pt} \bigg\{% 
	\hspace{0.5pt} {#1} \hspace{0.5pt} \bigg\} \hspace{1pt}}
\newcommand{\zzzz}[1]{ \hspace{1pt} \bigg\langle 
	\hspace{0.5pt} {#1} \hspace{0.5pt} \bigg\rangle \hspace{1pt}}


% REFERENCES & LABELS %

\makeatletter
\@ifpackageloaded{harvard}{}
  {\newcommand{\citeasnoun}{\cite}}
\makeatother

\newcommand{\ci}{\citeasnoun}
\newcommand{\cic}[2]{\citeasnoun[#1]{#2}}

\newcommand{\hci}[2]{ \href{#2}{\ci{#1}} }
\newcommand{\hcic}[3]{ \href{#3}{\cic{#1}{#2}} }
\newcommand{\hcite}[2]{ \href{#2}{\cite{#1}} }

\newcommand{\req}[1]{(\ref{#1})}






% BIBLIO %

\usepackage{harvard}
\usepackage{hyperref}
\hypersetup{colorlinks=true, allcolors=bamboo}

% TABLES %

\usepackage{booktabs}

% ALGORITHMS %

\usepackage{float}
\floatstyle{ruled}
\newfloat{algorithm}{hpt}{lop}
\floatname{algorithm}{Algorithm}

\usepackage{algorithm}
\usepackage{algpseudocode}










%\newif\ifhide
%\hidetrue %\hidefalse
%\newcommand{\hide}[1]{\ifhide {} \else {#1} \fi} 



% SPACING/TEXT %

\newcommand{\s}[1][1]{\hspace{#1pt}}
\newcommand{\mst}{{\scalebox{0.6}[0.6]{\(\hspace{-1pt}{*}\)}}}

\newcommand{\tq}[1]{{\textquotedblleft #1\textquotedblright}}
\newcommand{\ts}[2]{{#1\textsubscript{#2}}}


% PROBABILITY %

\newcommand{\cid}{\overset{\scrD}{\rightarrow}}
\newcommand{\eqd}{\overset{\scrD}{=}}
\newcommand{\cip}{\overset{p}{\rightarrow}}
\newcommand{\cas}{\overset{a.s.}{\rightarrow}}
\newcommand{\sip}{\overset{p}{\sim}}
\newcommand{\sas}{\overset{a.s.}{\sim}}

\newcommand{\setst}{\mathbb} % set style
\newcommand{\oprst}{\textsc} % operator style

\newcommand{\inds}[1]{ \mbf{1}_{ \s[-0.5] {#1} }}
\newcommand{\indf}[1]{ \mbf{1}_{ \s[-0.5] \{#1\} }}

\newcommand{\var}{\oprst{var}}
\newcommand{\std}{\oprst{std}}
\newcommand{\cov}{\oprst{cov}}
\newcommand{\Var}{\oprst{Var}}
\newcommand{\Cov}{\oprst{Cov}}
\newcommand{\Std}{\oprst{SD}}
\newcommand{\MSE}{\oprst{MSE}}
\newcommand{\RMSE}{\oprst{RMSE} }
\newcommand{\Exp}{\oprst{E}\s[.5]}
\newcommand{\Prb}{\oprst{P}\s[.5]}
\newcommand{\Qrb}{\ensuremath{\oprst{Q}\s[.5]}}

% SETS / MATH $

\newcommand{\nn}{\bbN}
\newcommand{\rn}{\bbR}
\newcommand{\zn}{\bbZ}
\newcommand{\qn}{\bbQ}

\newcommand{\Img}{\mathrm{Im}}
\newcommand{\Rea}{\mathrm{Re}}
\newcommand{\im}{\mathrm{i}}


\newcommand{\bst}{ \s[1.5] | \s[1.5] }
\newcommand{\cst}{ \s[0.5] : \s[0.5] }
\newcommand{\sst}{ \s[2.0] ; \s[0.5] }


\newcommand{\mat}[1]{\mbf{#1}}
\newcommand{\ones}{\rme}
\newcommand{\tr}{\textsc{Tr}}   


% GREEK %

\newcommand{\eps}{\epsilon}


% LETTER STYLES %
\usepackage{pgffor}

\foreach \x in {A,B,...,Z,a,b,...,z} 
{
  \expandafter\xdef\csname cal\x\endcsname{\noexpand 
	\ensuremath{\noexpand\mathcal{\x}}}
  \expandafter\xdef\csname scr\x\endcsname{\noexpand 
	\ensuremath{\noexpand\mathscr{\x}}}
  \expandafter\xdef\csname bb\x\endcsname{\noexpand 
	\ensuremath{\noexpand\mathbb{\x}}}
  \expandafter\xdef\csname rm\x\endcsname{\noexpand 
	\ensuremath{\noexpand\mathrm{\x}}}
  \expandafter\xdef\csname bf\x\endcsname{\noexpand 
	\ensuremath{\noexpand\mbf{\x}}}
}

% ARROWS %

\newcommand{\upto}{ \uparrow }
\newcommand{\downto}{ \downarrow }
\newcommand{\nto}{ \nrightarrow }



% COLABORATION %


\newcommand{\lrgtext}[1]{\oystertext{#1 -- lisa}}
\newcommand{\adstext}[1]{\bambootext{#1 -- alex}}
\newcommand{\agptext}[1]{\downytext{#1}}


\newcommand{\lrglmn}[1]{{\reversemarginpar \setstretch{0.64} %
	\marginnote{{\scriptsize 
	\oystertext{\emph{#1}--lisa}}}} }
\newcommand{\lrgrmn}[1]{{\normalmarginpar \setstretch{0.64} %
	\marginnote{{\scriptsize 
	\oystertext{\emph{#1}--lisa}}}} }

\newcommand{\adslmn}[1]{{\reversemarginpar \setstretch{0.64} %
	\marginnote{{\scriptsize 
	\bambootext{\emph{#1}--alex.s}}}} }
\newcommand{\adsrmn}[1]{{\normalmarginpar \setstretch{0.64} %
	\marginnote{{\scriptsize 
	\bambootext{\emph{#1} --alex.s}}}} }

\newcommand{\agplmn}[1]{{\reversemarginpar \setstretch{0.64} %
	\marginnote{{\scriptsize 
	\downytext{\emph{#1}--alex.p}}}}  }
\newcommand{\agprmn}[1]{{\normalmarginpar \setstretch{0.64} %
	\marginnote{{\scriptsize 
	\downytext{\emph{#1}--alex.p}}}}  }





% LOCAL %

\newcommand{\nv}{p}
\newcommand{\no}{n}
\newcommand{\tix}{j}
\newcommand{\six}{i}




\newcommand{\nb}{b}
\newcommand{\nh}{h}
\newcommand{\nz}{z}

\newcommand{\rrv}{Y}
\newcommand{\frv}{X}
\newcommand{\srv}{Z}

\newcommand{\ret}{\rmY}
\newcommand{\fet}{\rmX}
\newcommand{\set}{\rmZ}

\newcommand{\red}{\mbf{Y}}
\newcommand{\zed}{\mbf{Z}}

\newcommand{\bsig}{{\bm{\Sigma}}}
\newcommand{\bdel}{{\bm{\Delta}}}

\newcommand{\sphp}{\bbS^{\no-1}}
\newcommand{\covb}{{\bm{\Sigma}}}
\newcommand{\tfvol}{{\sigma}}
\newcommand{\tfv}{{\sigma^2}} 
\newcommand{\tsvol}{{\delta}}
\newcommand{\tsv}{{\delta^2}} 

\newcommand{\tfvolh}{\hat{\sigma}}
\newcommand{\tsvolh}{\hat{\delta}}
\newcommand{\tsvh}{{\hat{\delta}^2}} 
\newcommand{\tfvh}{{\hat{\sigma}^2}} 

\newcommand{\ter}{\scrT}
\newcommand{\vfr}{\scrR}
\newcommand{\vol}{\rmV}


\newcommand{\scov}{\mat{S}}
\newcommand{\covh}{\bm{\hat{\Sigma}}}
\newcommand{\efvol}{\ensuremath{\hat \sigma}}
\newcommand{\efv}{\ensuremath{\hat{\sigma}^2}} 
\newcommand{\esvol}{\ensuremath{\hat \delta}}
\newcommand{\esv}{\ensuremath{\hat{\delta}^2}} 
\newcommand{\esvi}{\ensuremath{\hat{\delta_i}^2}} 


\newcommand{\err}{\scrE}
\newcommand{\spar}{\tau}
\newcommand{\vpar}{t}
\newcommand{\covc}{\bm{\Sigma}_\spar}

%\newcommand{\minw}{w_{\scalebox{0.6}[0.6]{\(\hspace{-1pt}{h}\)}}}
%\newcommand{\optw}{w_{\scalebox{0.6}[0.6]{\(\hspace{-1pt}{b}\)}}}

\newcommand{\minw}{\hat{w}}
\newcommand{\optw}{w}
\newcommand{\eqww}{w_{\rme}}


\newcommand{\diag}{\textbf{diag}}    % factor exposure

\newcommand{\prp}[2]{\langle #1, #2 \rangle_{\hspace{-1pt}\nv}}
\newcommand{\prj}[2]{\langle #1, #2 \rangle}
\newcommand{\pri}[2]{\langle #1, #2 \rangle_\infty}
\newcommand{\ang}[2]{\theta_{\prj{#1}{#2}}}
\newcommand{\anp}[2]{\theta_{\prp{#1}{#2}}}
\newcommand{\Ang}[2]{\Theta_{\prj{#1}{#2}}}
\newcommand{\Anp}[2]{\Theta_{\prp{#1}{#2}}}


\newcommand{\chb}{\psi}
\newcommand{\eig}{\rms^2_{\hspace{-1pt} \nv}}
\newcommand{\blk}{\ell^2_{\hspace{-1pt} \nv}}
\newcommand{\sig}{\rms_\nv}

\newcommand{\me}{\mu}
\newcommand{\mei}{\mu_\infty}
\newcommand{\mep}{\mu_\nv}
\newcommand{\cv}{\rmd}
\newcommand{\cvp}{\rmd_\nv}
\newcommand{\cvi}{\rmd_\infty}



\newcommand{\llns}{\eta}
\newcommand{\svec}{\varphi}
\newcommand{\sy}{\chi_\nv}
\newcommand{\syk}{\chi_{\nv_k}}
\newcommand{\sign}{\sigma^2_{\rmX}}
\newcommand{\sigb}{\tfvol_{\hspace{-1.16pt} \nv}^2}
\newcommand{\sigh}{\tfvolh_{\hspace{-1.16pt} \nv}^2}

\newcommand{\thr}{\rho}
\newcommand{\roh}{\kappa}

%%%%% Commands Alex B %%%%%%%%%%


\newcommand{\ga}{\alpha}
\newcommand{\gb}{\beta}
\newcommand{\gc}{y}
\newcommand{\gd}{\delta}
\newcommand{\gf}{\phi}
\newcommand{\gl}{\lambda}
\newcommand{\gk}{\kappa}
\newcommand{\go}{\omega}
\newcommand{\gt}{\theta}
\newcommand{\gr}{\rho}
\newcommand{\gs}{\sigma}

\newcommand{\Gf}{\Phi}
\newcommand{\Go}{\Omega}
\newcommand{\Gc}{\Gamma}
\newcommand{\Gth}{\theta}
\newcommand{\Gd}{\Delta}
\newcommand{\Gs}{\Sigma}
\newcommand{\Gl}{\Lambda}
\renewcommand{\gc}{\gamma}
\newcommand{\mrm}{\mathrm}
\newcommand{\geps}{\varepsilon}

\newcommand{\T}{\top}

\newcommand{\paren}[1]{\left(#1\right)}

\newcommand{\ncal}{\mathcal{N}}


%Calculus
\newcommand{\diff}[1]{\frac{\partial }{ \partial {#1}}}


% Bamboo pallete %

\definecolor{deluge}{RGB}{124, 113, 173}
\definecolor{bamboo}{RGB}{220, 92, 5}
\definecolor{yellow}{RGB}{255, 172, 0}
\definecolor{orange}{RGB}{255, 144, 0}
\definecolor{oyster}{RGB}{151, 139, 125}
\definecolor{coral}{RGB}{199, 186, 167}
\definecolor{downy}{RGB}{110, 197, 184}
\definecolor{blueberry}{HTML}{6B7A8F}
\definecolor{alexbcolor}{HTML}{6088C7}

\def\delugetext#1{{\color{deluge}{{#1}}\color{deluge}}}
\def\bambootext#1{{\color{bamboo}{{#1}}\color{bamboo}}}
\def\yellowtext#1{{\color{yellow}{{#1}}\color{yellow}}}
\def\orangetext#1{{\color{orange}{{#1}}\color{orange}}}
\def\oystertext#1{{\color{oyster}{{#1}}\color{oyster}}}
\def\coraltext#1{{\color{coral}{{#1}}\color{coral}}}
\def\downytext#1{{\color{downy}{{#1}}\color{downy}}}






%\newcommand{\ind}[1]{{1}_{\{#1\}}}
\begin{document}

%\renewcommand{\labelenumi}{(\arabic{enumi})}
% Enable bold sum and product symbols
\ifxetex
  \let\lsum\sum
  \renewcommand{\sum}{\bm{\lsum}} 

  \let\lprod\prod
  \renewcommand{\prod}{\bm{\lprod}}
\else
\fi


\title{\vspace{-0.16in}{\bf \Large  
Efficient Portfolio}}

\date{\vspace{-0.32in}\today}

\newcommand{\thr}{\theta}
\newcommand{\gfn}{G}
\newcommand{\bB}{\mat{B}}
\newcommand{\bQ}{\mat{Q}}
\newcommand{\bV}{\mat{V}}
\newcommand{\diag}{\textbf{diag}}

\newcommand{\rme}{\mathrm{e}}
\newcommand{\rma}{\mathrm{a}}
\newcommand{\indm}{\ensuremath{\mathbf{1}_0}}
\newcommand{\ef}{f}
\newcommand{\vt}{\vartheta}
\newcommand{\lscr}{\mathscr{L}}

%\newcommand{\gd}{\delta}
%\newcommand{\gs}{\sigma}
%\newcommand{\gt}{\theta}
%\newcommand{\gb}{\beta}
%\newcommand{\ncal}{\mathcal{N}}
%\newcommand{\Gth}{\theta}

%\newcommand{\alex}[1]{\textcolor{red}{#1}}
%\newcommand{\ucal}{\mathcal{U}}
\newcommand{\tv}{\vartheta}
\newcommand{\zv}{z}


\maketitle

\vspace{-0.16in}
\begin{theorem}

Fix $\bQ = \sigma^2 \gb \gb^\top + \delta^2 \mat{I}$
for $\beta \in \bbr^p$, and $\sigma,\delta > 0$.  The quadratic program
\begin{align}
\begin{aligned}
 \hspace{0.5in}\max_{x \in \bbR^p} \ptt
  &\frac{1}{2} \pt \ip{x}{\bQ x} \\
 \text{subject to: }
	 &\textstyle\sum\nolimits_k \hpt x_k =1  \\
	 & x_k \geq 0
\label{con}
\end{aligned}
\end{align}
has a unique solution given by
\begin{align}
x_i = \frac{\paren{\eta-\gb_i}_+}{\sum_{i: \gb_i < \eta} \paren{\eta-\gb_i}} = \frac{\paren{\eta-\gb_i}_+}{\sum_{i=1}^p \paren{\eta-\gb_i}_+}
\end{align}
where $\eta$ is found as the solution to the following fixed-point problem:
\begin{align}
\eta = \frac{1/\gs^2+ \sum_{i:\eta>\gb_i  } \gb_i^2/\gd^2}{\sum_{i:\eta>\gb_i} \gb_i/\gd^2}
\end{align}
\end{theorem}
\begin{proof}
First, note that Problem \req{con} is equivalent to the following optimization problem:
\begin{align*}
\begin{aligned}
 \hspace{0.5in}\max_{v \in \bbR^p} \ptt
  &\frac{1}{2} \pt \ip{v^2}{\bQ v^2} \\
 \text{subject to: }
	&|v| = \ip{v}{v}= 1
\label{con2}
\end{aligned}
\end{align*}
where $v^2$ is considered to be elementwise square of the vector $v$, i.e. for $v \in \bbr^p$, $v^2 = \diag(v)v$.  This transformation implicitly includes the positivity constraint into the problem's formulation.  We can then form the Lagrangian with respect to $v$ and a lagrange multiplier, $\gl$:
\begin{align}
\lscr(v, \gl) = \gs^2 \paren{\sum_{i=1}^p \paren{v_i^2 \gb_i}}^2 + \gd^2 \sum_{i=1}^p v_i^4 + 2\gl\paren{\sum_{i=1}^p v_i^2 -1 }
\end{align}
Applying the KKT Criterion, we note that both our objective and constraints are convex, and therefore admit a unique solution.  Taking the derivative of $\lscr$ with respect to an arbitrary $v_i$ gives us:
\begin{align}
\diff  v_i \lscr(v,\gl) = 4\gs^2 \paren{\sum_{i=1}^p \paren{v_i^2\gb_i}}v_i\gb_i + 4 \gd^2 v_i^3 + 4 \gl v_i
\end{align}
For notational simplicity, define $$G(v) =\gb^\T v^2= \sum_{i=1}^p v_i^2 \gb_i$$
Our first order condition is therefore that, for every $i \in \{ 1, \ldots,p\}$:
\begin{align}\label{foc}
\gs^2 G(v) \gb_iv_i+  \gd^2  v_i^3 +  \gl v_i=0
\end{align}
Note that if $v_i^2>0 (\iff v_i \neq 0)$, we have:
\begin{align}
\gs^2 G(v) \gb_i + \gd^2 v_i^2 + \gl = 0
\end{align}
Furthermore, the conditions \req{foc} is trivially satisfied if $v_i=v_i^2=0$.  We therefore only consider the $v_i^2>0$, and for those elements, the following equations must be satisfied:
\begin{align}
\label{kkt1}\tag{KKT1} &\gs^2 G(v) \gb_i +\gd^2 v_i^2 + \gl=0\\
\label{kkt2} \tag{KKT2}&\sum_{i=1}^p v_i^2 =1
\end{align}
(\textit{Note that summing over all the elements of $v^2$ is equivalent to summing over only the nonzero elements.})\\
Let $w$ be another arbitrary $p-$vector.  If every $w_i^2>0$ satisfies
\begin{align}
\label{kktw}\tag{KKTw}  \gs^2 G(w) \gb_i + \gd^2 w_i^2  + \gl_w =0
\end{align}
where $ \gl_w = \gl \sum_{i=1}^p w_i^2$, then we have that 
\req{kkt1} and \req{kkt2} are satisfied by any $v_i^2$ defined as:
\begin{align}
v_i^2  = \frac{w_i^2}{\sum_{i=1}^p w_i^2}
\end{align}
This allows us just so solve \textit{only} \req{kktw}, and then normalize the solution.\\
Define:
\begin{align*}
\theta = \frac{\gs^2 G(w)}{\gl_w}
\end{align*}
Solving \req{kktw} for $w_i^2$ gives us:
\begin{align}\label{solw}
w_i^2 &= \frac{\gl_w}{\gd^2}\paren{1-\theta \gb_i}
\end{align}
Note that $w_i^2$ is either positive or $0$, so 
\begin{align}
G(w) = \sum_{i=1}^p w_i^2 \gb_i = \sum_{i:w_i^2>0} w_i^2 \gb_i
\end{align}
Therefore, multiplying each $w_i^2$ by $\gb_i$ and summing gives us:
\begin{align}
G(w) = \frac{\gl_w}{\gd^2}\paren{ \sum_{i:w_i^2>0} \gb_i - \theta \sum_{i:w_i^2>0} \gb_i^2}
\end{align}
and therefore
\begin{align}
\begin{aligned}
\theta &= \frac{\gs^2}{\gd^2}\paren{ \sum_{i:w_i^2>0} \gb_i - \theta \sum_{i:w_i^2>0} \gb_i^2}\\
&=\frac{\sum_{i:w_i^2>0} \gb_i/\gd^2}{1/\gs^2 + \sum_{i:w_i^2>0} \gb_i^2/\gd^2}\label{soltheta}
\end{aligned}
\end{align}
Also, note that from \req{kktw} that because $\gl_w>0$ (From KKT conditions and $\sum_{i=1}^p w_i^2>0$) and $\gd^2>0$, 
$$w_i^2>0 \iff 1-\theta \gb_i>0.$$
We therefore find the condition that 
\begin{align}
\theta \beta_i < 1.
\end{align}
Again, for simplicity, define $\eta= 1/\theta$
Therefore, the summand condition that $w_i^2>0$ is equivalent $\eta > \gb_i$.  Therefore \req{soltheta}, rewritten in terms of $\eta$, becomes:
\begin{align}
\eta &=\frac{\gl_w}{\gs^2G(w)}\\
&= \frac{1/\gs^2 + \sum_{i:\eta>\gb_i} \gb_i^2 / \gd^2}{\sum_{i:\eta>\gb_i}\gb_i / \gd^2} \label{etasol}
\end{align}
This is now an explicit fixed-point equation over the summed terms; we can therefore find the solution by simply ordering the elements $\gb_i$ and including them until we find the first $\gb_i$  such that the righthand side of the expression is greater than the lefthand side.

Note that because $w_i^2 \geq 0$, we have
\begin{align*}
w_i^2 = \frac{\gs^2 G(w)}{\gd^2}\paren{\eta - \gb_i}_+
\end{align*}
We solve for $\eta$ in the fixed point equation, so we know what elements $w_i^2$ are 0.  Further, we note that
\begin{align}
\sum_{i:w_i^2>0} w_i^2 = \gs^2G(w)\sum_{i:\gb_i<\eta} \frac{\paren{\eta-\gb_i}}{\gd^2}
\end{align}
And therefore
\begin{align}
v_i^2 &= \frac{w_i}{\sum_{i:w_i^2>0}w_i^2} \\
&= \paren{\frac{\gs^2 G(w)}{\gd^2}\paren{\eta-\gb_i}_+} \frac{1}{\gs^2G(w)\sum_{i:\gb_i<\eta} \frac{\paren{\eta-\gb_i}}{\gd^2}}\\
&= \frac{\paren{\eta-\gb_i}_+}{\sum_{i:\gb_i<\eta}\paren{\eta-\gb_i}}
\end{align}
with $\eta$ as found as in \req{etasol}.
We therefore have
\begin{align}
v_i^2  = x_i &= \frac{\paren{\eta-\gb_i}_+}{\sum_{i: \gb_i<\eta} \paren{\eta-\gb_i}}\\
&= \frac{\paren{\eta-\gb_i}_+}{\sum_{i=1}^p \paren{\eta-\gb_i}_+}
\end{align}

\end{proof}

\end{document}






