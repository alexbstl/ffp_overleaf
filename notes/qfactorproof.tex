\documentclass[12pt,leqno,letterpaper]{article}


% PAGE SETTINGS %

\parskip 0in
\parindent 0.32in
%\voffset -0.125in

\usepackage[letterpaper]{geometry}
\geometry{text={5.562306in,9in}, centering} % golden ratio :)
%\renewcommand{\baselinestretch}{1.02} 

% FONTS %

\usepackage{ifxetex}
\ifxetex
  \usepackage[T1]{fontenc}
  \usepackage[no-math]{fontspec}

  \setmainfont[Scale=1.0]{Times NR MT Std}

  \usepackage[cal=euler, calscaled=1,
	      scr=boondox, scrscaled=1]{mathalfa}

  \usepackage[mtpcal,mtphrb,mtpfrak,noamssymbols,nofontinfo,
  	      subscriptcorrection,zswash]{mtpro2}

  %\DeclareMathSizes{12}{13}{9}{7}
\else
  \usepackage{lmodern}
  \usepackage{amsfonts}
  \usepackage{amssymb}
  \usepackage{mathrsfs}
  % to match mt2pro
  \newcommand{\mbf}{\mathbf} 
  \newcommand{\what}[1]{\widehat{#1}}
\fi

% THEOREMS %

\usepackage{amsthm}

\newtheoremstyle{mytheoremstyle}
{\topsep}                    % Space above
{\topsep}                    % Space below
{\itshape}                   % Body font
{}                           % Indent amount
{\bfseries}                  % Theorem head font
{.}                          % Punctuation after theorem head
{.5em}                       % Space after theorem head
{}  % Theorem head spec (can be left empty, meaning ‘normal’)

\theoremstyle{mytheoremstyle}

\makeatletter
\@ifclassloaded{report}{
   \newtheorem{theorem}{Theorem} }
{  \newtheorem{theorem}{Theorem}[section] }
\makeatother

\newtheorem{thm}[theorem]{Theorem}
\newtheorem{proposition}[theorem]{Proposition}
\newtheorem{lemma}[theorem]{Lemma}
\newtheorem{corollary}[theorem]{Corollary}
\newtheorem{definition}[theorem]{Definition}
\newtheorem{remark}[theorem]{Remark}
\newtheorem{example}[theorem]{Example}
\newtheorem{condition}[theorem]{Condition}
\newtheorem{assumption}[theorem]{Assumption}

\newtheorem*{theo}{Theorem}
\newtheorem*{lemm}{Lemma}
\newtheorem*{defn}{Definition}
\newtheorem*{prop}{Proposition}
\newtheorem*{assm}{Assumption}
\newtheorem*{cond}{Condition}
\newtheorem*{remk}{Remark}
\newtheorem*{expl}{Example}


% REFERENCES %

\usepackage{harvard}

% SECTIONS %

\usepackage{titlesec}

\titleformat{\section}[runin]
{\normalfont\bfseries}
{\thesection.}{0.08in}{}[.]

\titlespacing{\section}{\parindent}{0.08in}{0.08in}

\titleformat{\subsection}[runin]
{\normalfont\slshape}
{\thesubsection.}{0.04in}{}[.]

\titlespacing{\subsection}{\parindent}{0.08in}{0.08in}


% ABSTRACT

\renewcommand{\abstractname}{\vspace{-0.48in}}

\renewenvironment{abstract}
{\small
\begin{center}
  \bfseries \abstractname\vspace{-.5em}\vspace{0pt}
\end{center}
\list{}{
	\setlength{\leftmargin}{\parindent}
    \setlength{\rightmargin}{\leftmargin}
    \setlength{\parsep}{0in}
	\itemindent \parindent} 
  \item\relax}
{\endlist}

% TABLES AND FIGURES


\usepackage{caption}
\captionsetup[figure]{margin=0.32in,
	labelsep=period,labelfont=bf,textfont=it}
\captionsetup[table]{margin=0.32in,
	labelsep=period,labelfont=bf,textfont=it}





% BASIC PACKAGES %

\usepackage{makecell}
\usepackage{graphicx}
\usepackage{enumitem}
\usepackage{booktabs}
\usepackage{framed}
\usepackage{multirow}
\usepackage{scrextend}
\usepackage{stackengine}
\usepackage{subcaption}
\usepackage{comment}
\usepackage{url}
\usepackage{bm}
\usepackage[bottom]{footmisc}

% MARGIN NOTES %

\usepackage{marginnote}
\usepackage{setspace}

\newcommand{\lmn}[1]{{\reversemarginpar \setstretch{0.64} %
	\marginnote{{\scriptsize \bambootext{\emph{#1}}}}} }

\newcommand{\rmn}[1]{{\normalmarginpar \setstretch{0.64} %
	\marginnote{{\scriptsize \bambootext{\emph{#1}}}}} }

\setlength\marginparwidth{64pt}

% RAISE/LOWER %

\makeatletter
\newcommand{\raisemath}[1]{\mathpalette{\raisem@th{#1}}}
\newcommand{\raisem@th}[3]{\raisebox{#1}{$#2#3$}}
\makeatother

% HIDE MACRO %

\newif\ifhide
\hidetrue %\hidefalse
\newcommand{\hide}[1]{\ifhide {} \else {#1} \fi} 

% EMPHASIZE BOX %

\usepackage{empheq}
\newcommand{\widefbox}[1]{\fbox{\hspace{0.33in}#1\hspace{0.33in}}}


% CHAR MAPS: e.g. \calS for \mathcal{S}  %

\usepackage{pgffor}

\foreach \x in {A,...,Z}{%
  \expandafter\xdef\csname cal\x\endcsname{\noexpand %
	\ensuremath{\noexpand\mathcal{\x}}}
  \expandafter\xdef\csname scr\x\endcsname{\noexpand %
	\ensuremath{\noexpand\mathscr{\x}}}
  \expandafter\xdef\csname bb\x\endcsname{\noexpand %
	\ensuremath{\noexpand\mathbb{\x}}}
  \expandafter\xdef\csname rm\x\endcsname{\noexpand %
	\ensuremath{\noexpand\mathrm{\x}}}
  \expandafter\xdef\csname bf\x\endcsname{\noexpand %
	\ensuremath{\noexpand\mathbf{\x}}}
}

% ACCENTS %

\newcommand{\wh}[1]{\widehat{#1}}
\newcommand{\wt}[1]{\widetilde{#1}}

% SPACING %
\newcommand{\hs}[1]{ \hspace{#1pt} }

\newcommand{\mptt}{\hspace{-2pt}  }
\newcommand{\mpt}{ \hspace{-1pt}  }
\newcommand{\mhpt}{\hspace{-0.5pt}}
\newcommand{\hpt}{ \hspace{0.5pt} }

\newcommand{\pt}{ \hspace{1pt}  }
\newcommand{\ptt}{ \hspace{2pt} }
\newcommand{\pttt}{ \hspace{3pt}}
\newcommand{\ptttt}{\hspace{4pt}}

% BASIC MATH %

%\newcommand{\bst}{ \hspace{1.5pt} | \hspace{1.5pt} }
%\newcommand{\cst}{ \hspace{0.5pt} : \hspace{0.5pt} }
%\newcommand{\sst}{ \hspace{2pt} ; \hspace{0.5pt} }
\newcommand{\evl}{ \left| \right. }

\newcommand{\abs}[1]{\left| {#1} \right|}
\newcommand{\ceil}[1]{\left\lceil #1 \right\rceil}
\newcommand{\floor}[1]{\left\lfloor #1 \right\rfloor}

% ARROWS %

%\newcommand{\upto}{ \uparrow }
%\newcommand{\downto}{ \downarrow }
%\newcommand{\nto}{ \nrightarrow }

% TEXT %

%\newcommand{\tq}[1]{{\textquotedblleft #1\textquotedblright}}
%\newcommand{\ts}[2]{{#1\textsubscript{#2}}}


% Brackets, Parentheses, etc %

\newcommand{\pa}[1]{\hspace{1pt} \left( \hspace{0.5pt} %
	{#1} \hspace{0.5pt} \right) \hspace{1pt}}
\newcommand{\qa}[1]{\hspace{1pt} \left[ \hspace{0.5pt} %
	{#1} \hspace{0.5pt} \right] \hspace{1pt}}
\newcommand{\fa}[1]{\hspace{1pt} \left \{ \hspace{0.5pt} % 
	{#1} \hspace{0.5pt} \right \}\hspace{1pt}}
\newcommand{\za}[1]{\hspace{1pt} \left\langle \hspace{0.5pt} %
	{#1} \hspace{0.5pt} \right\rangle \hspace{1pt}}

\newcommand{\p}[1]{ \hspace{1pt} ( \hspace{0.25pt} %
	{#1} \hspace{0.25pt} ) \hspace{1pt}}
\newcommand{\q}[1]{ \hspace{1pt} [ \hspace{0.25pt} %
	{#1} \hspace{0.25pt} ] \hspace{1pt}}
\newcommand{\f}[1]{ \hspace{1pt} \{ \hspace{0.25pt} %
	{#1} \hspace{0.25pt} \} \hspace{1pt}}
\newcommand{\z}[1]{ \hspace{1pt} \langle \hspace{0.25pt} %
	{#1} \hspace{0.25pt}\rangle\hspace{1pt}}
\newcommand{\g}[1]{ \hspace{1pt} \langle \hspace{0.25pt} %
	{#1} \hspace{0.25pt}\rangle\hspace{1pt}}

\newcommand{\pp}[1]{ \hspace{1pt} \big( \hspace{0.5pt} %
	{#1} \hspace{0.5pt} \big) \hspace{1pt}}
\newcommand{\qq}[1]{ \hspace{1pt} \big[ \hspace{0.5pt} %
	{#1} \hspace{0.5pt} \big] \hspace{1pt}}
\newcommand{\ff}[1]{ \hspace{1pt} \big\{ \hspace{0.5pt} %
	{#1} \hspace{0.5pt} \big\} \hspace{1pt}}
\newcommand{\zz}[1]{ \hspace{1pt} \big\langle \hspace{0.5pt} %
	{#1} \hspace{0.5pt} \big\rangle \hspace{1pt}}

\newcommand{\ppp}[1]{ \hspace{1pt} \Big( \hspace{0.5pt} %
	{#1} \hspace{0.5pt} \Big) \hspace{1pt}}
\newcommand{\qqq}[1]{ \hspace{1pt} \Big[ \hspace{0.5pt} %
	{#1}  \hspace{0.5pt} \Big] \hspace{1pt}}
\newcommand{\fff}[1]{ \hspace{1pt} \Big\{ \hspace{0.5pt} %
	{#1}  \hspace{0.5pt} \Big\} \hspace{1pt}}
\newcommand{\zzz}[1]{ \hspace{1pt} \Big\langle \hspace{0.5pt} %
	{#1} \hspace{0.5pt} \Big\rangle \hspace{1pt}}

\newcommand{\pppp}[1]{ \hspace{1pt} \bigg(% 
	\hspace{0.5pt} {#1} \hspace{0.5pt} \bigg) \hspace{1pt}}
\newcommand{\qqqq}[1]{ \hspace{1pt} \bigg[% 
	\hspace{0.5pt} {#1} \hspace{0.5pt} \bigg] \hspace{1pt}}
\newcommand{\ffff}[1]{ \hspace{1pt} \bigg\{% 
	\hspace{0.5pt} {#1} \hspace{0.5pt} \bigg\} \hspace{1pt}}
\newcommand{\zzzz}[1]{ \hspace{1pt} \bigg\langle 
	\hspace{0.5pt} {#1} \hspace{0.5pt} \bigg\rangle \hspace{1pt}}


% REFERENCES & LABELS %

\makeatletter
\@ifpackageloaded{harvard}{}
  {\newcommand{\citeasnoun}{\cite}}
\makeatother

\newcommand{\ci}{\citeasnoun}
\newcommand{\cic}[2]{\citeasnoun[#1]{#2}}

\newcommand{\hci}[2]{ \href{#2}{\ci{#1}} }
\newcommand{\hcic}[3]{ \href{#3}{\cic{#1}{#2}} }
\newcommand{\hcite}[2]{ \href{#2}{\cite{#1}} }

\newcommand{\req}[1]{(\ref{#1})}






% BIBLIO %

\usepackage{harvard}
\usepackage{hyperref}
\hypersetup{colorlinks=true, allcolors=bamboo}

% TABLES %

\usepackage{booktabs}

% ALGORITHMS %

\usepackage{float}
\floatstyle{ruled}
\newfloat{algorithm}{hpt}{lop}
\floatname{algorithm}{Algorithm}

\usepackage{algorithm}
\usepackage{algpseudocode}










%\newif\ifhide
%\hidetrue %\hidefalse
%\newcommand{\hide}[1]{\ifhide {} \else {#1} \fi} 



% SPACING/TEXT %

\newcommand{\s}[1][1]{\hspace{#1pt}}
\newcommand{\mst}{{\scalebox{0.6}[0.6]{\(\hspace{-1pt}{*}\)}}}

\newcommand{\tq}[1]{{\textquotedblleft #1\textquotedblright}}
\newcommand{\ts}[2]{{#1\textsubscript{#2}}}


% PROBABILITY %

\newcommand{\cid}{\overset{\scrD}{\rightarrow}}
\newcommand{\eqd}{\overset{\scrD}{=}}
\newcommand{\cip}{\overset{p}{\rightarrow}}
\newcommand{\cas}{\overset{a.s.}{\rightarrow}}
\newcommand{\sip}{\overset{p}{\sim}}
\newcommand{\sas}{\overset{a.s.}{\sim}}

\newcommand{\setst}{\mathbb} % set style
\newcommand{\oprst}{\textsc} % operator style

\newcommand{\inds}[1]{ \mbf{1}_{ \s[-0.5] {#1} }}
\newcommand{\indf}[1]{ \mbf{1}_{ \s[-0.5] \{#1\} }}

\newcommand{\var}{\oprst{var}}
\newcommand{\std}{\oprst{std}}
\newcommand{\cov}{\oprst{cov}}
\newcommand{\Var}{\oprst{Var}}
\newcommand{\Cov}{\oprst{Cov}}
\newcommand{\Std}{\oprst{SD}}
\newcommand{\MSE}{\oprst{MSE}}
\newcommand{\RMSE}{\oprst{RMSE} }
\newcommand{\Exp}{\oprst{E}\s[.5]}
\newcommand{\Prb}{\oprst{P}\s[.5]}
\newcommand{\Qrb}{\ensuremath{\oprst{Q}\s[.5]}}

% SETS / MATH $

\newcommand{\nn}{\bbN}
\newcommand{\rn}{\bbR}
\newcommand{\zn}{\bbZ}
\newcommand{\qn}{\bbQ}

\newcommand{\Img}{\mathrm{Im}}
\newcommand{\Rea}{\mathrm{Re}}
\newcommand{\im}{\mathrm{i}}


\newcommand{\bst}{ \s[1.5] | \s[1.5] }
\newcommand{\cst}{ \s[0.5] : \s[0.5] }
\newcommand{\sst}{ \s[2.0] ; \s[0.5] }


\newcommand{\mat}[1]{\mbf{#1}}
\newcommand{\ones}{\rme}
\newcommand{\tr}{\textsc{Tr}}   


% GREEK %

\newcommand{\eps}{\epsilon}


% LETTER STYLES %
\usepackage{pgffor}

\foreach \x in {A,B,...,Z,a,b,...,z} 
{
  \expandafter\xdef\csname cal\x\endcsname{\noexpand 
	\ensuremath{\noexpand\mathcal{\x}}}
  \expandafter\xdef\csname scr\x\endcsname{\noexpand 
	\ensuremath{\noexpand\mathscr{\x}}}
  \expandafter\xdef\csname bb\x\endcsname{\noexpand 
	\ensuremath{\noexpand\mathbb{\x}}}
  \expandafter\xdef\csname rm\x\endcsname{\noexpand 
	\ensuremath{\noexpand\mathrm{\x}}}
  \expandafter\xdef\csname bf\x\endcsname{\noexpand 
	\ensuremath{\noexpand\mbf{\x}}}
}

% ARROWS %

\newcommand{\upto}{ \uparrow }
\newcommand{\downto}{ \downarrow }
\newcommand{\nto}{ \nrightarrow }



% COLABORATION %


\newcommand{\lrgtext}[1]{\oystertext{#1 -- lisa}}
\newcommand{\adstext}[1]{\bambootext{#1 -- alex}}
\newcommand{\agptext}[1]{\downytext{#1}}


\newcommand{\lrglmn}[1]{{\reversemarginpar \setstretch{0.64} %
	\marginnote{{\scriptsize 
	\oystertext{\emph{#1}--lisa}}}} }
\newcommand{\lrgrmn}[1]{{\normalmarginpar \setstretch{0.64} %
	\marginnote{{\scriptsize 
	\oystertext{\emph{#1}--lisa}}}} }

\newcommand{\adslmn}[1]{{\reversemarginpar \setstretch{0.64} %
	\marginnote{{\scriptsize 
	\bambootext{\emph{#1}--alex.s}}}} }
\newcommand{\adsrmn}[1]{{\normalmarginpar \setstretch{0.64} %
	\marginnote{{\scriptsize 
	\bambootext{\emph{#1} --alex.s}}}} }

\newcommand{\agplmn}[1]{{\reversemarginpar \setstretch{0.64} %
	\marginnote{{\scriptsize 
	\downytext{\emph{#1}--alex.p}}}}  }
\newcommand{\agprmn}[1]{{\normalmarginpar \setstretch{0.64} %
	\marginnote{{\scriptsize 
	\downytext{\emph{#1}--alex.p}}}}  }





% LOCAL %

\newcommand{\nv}{p}
\newcommand{\no}{n}
\newcommand{\tix}{j}
\newcommand{\six}{i}




\newcommand{\nb}{b}
\newcommand{\nh}{h}
\newcommand{\nz}{z}

\newcommand{\rrv}{Y}
\newcommand{\frv}{X}
\newcommand{\srv}{Z}

\newcommand{\ret}{\rmY}
\newcommand{\fet}{\rmX}
\newcommand{\set}{\rmZ}

\newcommand{\red}{\mbf{Y}}
\newcommand{\zed}{\mbf{Z}}

\newcommand{\bsig}{{\bm{\Sigma}}}
\newcommand{\bdel}{{\bm{\Delta}}}

\newcommand{\sphp}{\bbS^{\no-1}}
\newcommand{\covb}{{\bm{\Sigma}}}
\newcommand{\tfvol}{{\sigma}}
\newcommand{\tfv}{{\sigma^2}} 
\newcommand{\tsvol}{{\delta}}
\newcommand{\tsv}{{\delta^2}} 

\newcommand{\tfvolh}{\hat{\sigma}}
\newcommand{\tsvolh}{\hat{\delta}}
\newcommand{\tsvh}{{\hat{\delta}^2}} 
\newcommand{\tfvh}{{\hat{\sigma}^2}} 

\newcommand{\ter}{\scrT}
\newcommand{\vfr}{\scrR}
\newcommand{\vol}{\rmV}


\newcommand{\scov}{\mat{S}}
\newcommand{\covh}{\bm{\hat{\Sigma}}}
\newcommand{\efvol}{\ensuremath{\hat \sigma}}
\newcommand{\efv}{\ensuremath{\hat{\sigma}^2}} 
\newcommand{\esvol}{\ensuremath{\hat \delta}}
\newcommand{\esv}{\ensuremath{\hat{\delta}^2}} 
\newcommand{\esvi}{\ensuremath{\hat{\delta_i}^2}} 


\newcommand{\err}{\scrE}
\newcommand{\spar}{\tau}
\newcommand{\vpar}{t}
\newcommand{\covc}{\bm{\Sigma}_\spar}

%\newcommand{\minw}{w_{\scalebox{0.6}[0.6]{\(\hspace{-1pt}{h}\)}}}
%\newcommand{\optw}{w_{\scalebox{0.6}[0.6]{\(\hspace{-1pt}{b}\)}}}

\newcommand{\minw}{\hat{w}}
\newcommand{\optw}{w}
\newcommand{\eqww}{w_{\rme}}


\newcommand{\diag}{\textbf{diag}}    % factor exposure

\newcommand{\prp}[2]{\langle #1, #2 \rangle_{\hspace{-1pt}\nv}}
\newcommand{\prj}[2]{\langle #1, #2 \rangle}
\newcommand{\pri}[2]{\langle #1, #2 \rangle_\infty}
\newcommand{\ang}[2]{\theta_{\prj{#1}{#2}}}
\newcommand{\anp}[2]{\theta_{\prp{#1}{#2}}}
\newcommand{\Ang}[2]{\Theta_{\prj{#1}{#2}}}
\newcommand{\Anp}[2]{\Theta_{\prp{#1}{#2}}}


\newcommand{\chb}{\psi}
\newcommand{\eig}{\rms^2_{\hspace{-1pt} \nv}}
\newcommand{\blk}{\ell^2_{\hspace{-1pt} \nv}}
\newcommand{\sig}{\rms_\nv}

\newcommand{\me}{\mu}
\newcommand{\mei}{\mu_\infty}
\newcommand{\mep}{\mu_\nv}
\newcommand{\cv}{\rmd}
\newcommand{\cvp}{\rmd_\nv}
\newcommand{\cvi}{\rmd_\infty}



\newcommand{\llns}{\eta}
\newcommand{\svec}{\varphi}
\newcommand{\sy}{\chi_\nv}
\newcommand{\syk}{\chi_{\nv_k}}
\newcommand{\sign}{\sigma^2_{\rmX}}
\newcommand{\sigb}{\tfvol_{\hspace{-1.16pt} \nv}^2}
\newcommand{\sigh}{\tfvolh_{\hspace{-1.16pt} \nv}^2}

\newcommand{\thr}{\rho}
\newcommand{\roh}{\kappa}

%%%%% Commands Alex B %%%%%%%%%%


\newcommand{\ga}{\alpha}
\newcommand{\gb}{\beta}
\newcommand{\gc}{y}
\newcommand{\gd}{\delta}
\newcommand{\gf}{\phi}
\newcommand{\gl}{\lambda}
\newcommand{\gk}{\kappa}
\newcommand{\go}{\omega}
\newcommand{\gt}{\theta}
\newcommand{\gr}{\rho}
\newcommand{\gs}{\sigma}

\newcommand{\Gf}{\Phi}
\newcommand{\Go}{\Omega}
\newcommand{\Gc}{\Gamma}
\newcommand{\Gth}{\theta}
\newcommand{\Gd}{\Delta}
\newcommand{\Gs}{\Sigma}
\newcommand{\Gl}{\Lambda}
\renewcommand{\gc}{\gamma}
\newcommand{\mrm}{\mathrm}
\newcommand{\geps}{\varepsilon}

\newcommand{\T}{\top}

\newcommand{\paren}[1]{\left(#1\right)}

\newcommand{\ncal}{\mathcal{N}}


%Calculus
\newcommand{\diff}[1]{\frac{\partial }{ \partial {#1}}}


% Bamboo pallete %

\definecolor{deluge}{RGB}{124, 113, 173}
\definecolor{bamboo}{RGB}{220, 92, 5}
\definecolor{yellow}{RGB}{255, 172, 0}
\definecolor{orange}{RGB}{255, 144, 0}
\definecolor{oyster}{RGB}{151, 139, 125}
\definecolor{coral}{RGB}{199, 186, 167}
\definecolor{downy}{RGB}{110, 197, 184}
\definecolor{blueberry}{HTML}{6B7A8F}
\definecolor{alexbcolor}{HTML}{6088C7}

\def\delugetext#1{{\color{deluge}{{#1}}\color{deluge}}}
\def\bambootext#1{{\color{bamboo}{{#1}}\color{bamboo}}}
\def\yellowtext#1{{\color{yellow}{{#1}}\color{yellow}}}
\def\orangetext#1{{\color{orange}{{#1}}\color{orange}}}
\def\oystertext#1{{\color{oyster}{{#1}}\color{oyster}}}
\def\coraltext#1{{\color{coral}{{#1}}\color{coral}}}
\def\downytext#1{{\color{downy}{{#1}}\color{downy}}}






%\newcommand{\ind}[1]{{1}_{\{#1\}}}
\begin{document}

%\renewcommand{\labelenumi}{(\arabic{enumi})}
% Enable bold sum and product symbols
\ifxetex
  \let\lsum\sum
  \renewcommand{\sum}{\bm{\lsum}} 

  \let\lprod\prod
  \renewcommand{\prod}{\bm{\lprod}}
\else
\fi


\title{\vspace{-0.16in}{\bf \Large  
Efficient Portfolio}}

\date{\vspace{-0.32in}\today}

\newcommand{\thr}{\theta}
\newcommand{\gfn}{G}
\newcommand{\bB}{\mat{B}}
\newcommand{\bQ}{\mat{Q}}
\newcommand{\bV}{\mat{V}}
\newcommand{\diag}{\textbf{diag}}
\newcommand{\be}{\mat{e}}


\newcommand{\rme}{\mathrm{e}}
\newcommand{\rma}{\mathrm{a}}
\newcommand{\indm}{\ensuremath{\mathbf{1}_0}}
\newcommand{\ef}{f}
\newcommand{\vt}{\vartheta}
\newcommand{\lscr}{\mathscr{L}}

%\newcommand{\gd}{\delta}
%\newcommand{\gs}{\sigma}
%\newcommand{\gt}{\theta}
%\newcommand{\gb}{\beta}
%\newcommand{\ncal}{\mathcal{N}}
%\newcommand{\Gth}{\theta}

%\newcommand{\alex}[1]{\textcolor{red}{#1}}
%\newcommand{\ucal}{\mathcal{U}}
\newcommand{\tv}{\vartheta}
\newcommand{\zv}{z}
\newcommand{\Bm}{\mat{B}}
\newcommand{\Vm}{\mat{V}}

\maketitle

\vspace{-0.16in}
\begin{theorem}
Let $p >>q$.
Fix $\bQ = \Bm \Vm \Bm^\top + \Gd$
for $\Bm \in \bbr^{p\times q}$, $\Vm = \mathrm{diag}(\gs_1^2, \ldots, \gs_q^2)$ and $\Gd= \diag(\gd_1^2,\ldots,\gd_p^2)$.  The quadratic program
\begin{align}
\begin{aligned}
 \hspace{0.5in}\max_{x \in \bbR^p} \ptt
  &\ip{x}{\bQ x} \\
 \text{subject to: }
	 &\textstyle\sum\nolimits_i \hpt x_i =1  \\
	 & x_i \geq 0
\label{con}
\end{aligned}
\end{align}
has a unique solution given by
\begin{align}
x_i&=\frac{1}{\gd_i^2}\paren{1 - \sum_{k=1}^q \theta_k \gb_i^k}_+ \paren{ \sum_{k:i_k>0} \frac{1}{\gd_i^2}\paren{1 - \sum_{k=1}^q \theta_k \gb_i^k}}^{-1}
\end{align}
where $\theta$ is found as the solution to the following fixed-point problem:
\begin{align}
\mat{A}_\vartheta \vartheta = \mat{b}_\vartheta
\end{align}
as specified in \req{linsys}
\end{theorem}
\begin{proof}
Our Lagrangian for this problem is
\begin{align}
\lscr(x,\gl,\ell_1, \ldots, \ell_p) = x^\T \paren{\mat{BVB^\T} +\Gd}x + 2 \gl\paren{1-\be^\T x} +\sum_{i=1}^p x_i \ell_i 
\end{align}
Taking the Gradient with respect to $x$ gives us:
\begin{align}
\grad_x \lscr = 2\paren{\mat{BVB^\T}}x + 2 \mat{\Gd}x - 2 \gl \be + \ell = 0
\end{align}
where $\ell \in \bbr^p$.  Note that our complementary slackness condition states that $\ell_i x_i =0$ for all $i$.  Therefore, when $x_i \neq 0$, $\ell_i=0$.  Let 
\begin{align*}
\chi_{\{x = 0\}} = [\inds{\{x_1 = 0\}}, \ldots , \inds{\{x_p = 0\}}]^\T
\end{align*}
Our first order conditions are therefore:
\begin{align}
\label{kkt1} \tag{kkt1}&\begin{aligned}
 2\paren{\mat{BVB^\T}}x +  2\mat{\Gd}x - 2 \gl \be + \ell \chi_{\{x = 0\}}=0
 \end{aligned}\\
\label{kkt2} \tag{kkt2}&1 - \be^\T x = 0
\end{align}
Let $\be_i$ be the vector of all zero's except in the $i$\textsuperscript{th} location.  Note our first order condition implies that if $x_i = 0$, we have the following:
\begin{align}
\mat{BVB^\T}\textbf{diag}(\be_i)x +  \Gd \textbf{diag}(\be_i)x - 2\gl + \ell_i = -2\gl + \ell_i =0
\end{align} 
and we therefore have that, in this case, $\ell_i = 2 \gl$.\\  For simplicity, we introduce the following notation:
\begin{align*}
\mat{B} &= [ \gb^1, \, \gb^2, \ldots, \gb^q]\\ 
\text{for } u \in \bbr^p, \; G_k(\mat{u}) &=  {\gb^k}^\T u = \sum_{i=1}^p \gb^k_i u_i
\end{align*}
Combining all these together, define
\begin{align}
G(u) = \mat{B^\T u} = G(\mat{u})
\end{align}
Therefore
\begin{align}
\mat{BVB^\T}x &= \mat{B}\paren{\mat{VB^\T}x} = \mat{BV}G(\mat{x})\\
&=\mat{B} \begin{pmatrix}
\gs_1^2 G_1(x) \\
\gs_2^2 G_2(x)\\
\vdots\\
\gs_q^2 G_q(x)
\end{pmatrix}\\
&= \begin{pmatrix}
\sum_{k=1}^q \gb^k_1 \gs_k^2 G_k(x) \\
\sum_{k=1}^q \gb^k_2 \gs_k^2 G_k(x) \\
\vdots\\
\sum_{k=1}^q \gb^k_p \gs_k^2 G_k(x) 
\end{pmatrix}\\
\end{align}
Therefore the $i$\textsuperscript{th} row of \req{kkt1}, assuming $x_i \neq 0$ is:
\begin{align}
0&=\sum_{k=1}^q \gs^2_k G_k(x) \gb^k_i + \gd_i^2 x_i - \gl 
\end{align}
Let $w \in \bbr^p$ such that each element $w_i \geq 0$.  If for each element $w_i \neq 0$, solves the modified first order kkt condition:
\begin{align}\label{kktw}
\tag{kktw} \sum_{k=1}^q \gs_k^2 G_k(w) \gb_i^k +\gd_i^2 w_i - \gl_w = 0
\end{align}
where $\gl_w = \gl\sum_{i=1}^p w_i$
then the vector $x$ with elements
\begin{align}
x_i = \frac{w_i}{\sum_{j=1}^p w_j}
\end{align}
solves \req{kkt1} and \req{kkt2}.  Solving for $w_i$ gives us:
\begin{align}
w_i = \frac{1}{\gd_i^2}\paren{\gl_w - \sum_{k=1}^q \gs_k^2 G_k(w) \gb_i^k}_+
\end{align}
Defining
\begin{align} \label{thetadef}
\theta_k = \frac{\gs_k^2 G_k(w)}{\gl_w}
\end{align}
we can rewrite
\begin{align}
w_i = \frac{\gl_w}{\gd_i^2}\paren{1 - \sum_{k=1}^q \theta_k \gb_i^k}_+
\end{align}
where the positivity constraint comes from the fact that if $\gl(w)> \sum_{k=1}^q \theta_k \gb_i^k$, then according to \req{kktw}, $w_i<0$, which is not allowed, and so $w_i=0$ is the only solution that satisfies the original KKT conditions.  We therefore find that
\begin{align}
x_i &= \frac{\gl_w}{\gd_i^2}\paren{1 - \sum_{k=1}^q \theta_k\gb_i^k}_+ \paren{ \gl_w\sum_{i=1}^p\frac{1}{\gd_i^2}\paren{1 - \sum_{k=1}^q \theta_k \gb_i^k}_+}^{-1}\\
&=\frac{1}{\gd_i^2}\paren{1 - \sum_{k=1}^q \theta_k \gb_i^k}_+ \paren{ \sum_{i:w_i>0} \frac{1}{\gd_i^2}\paren{1 - \sum_{k=1}^q \theta_k \gb_i^k}}^{-1}\\
(x_i >0 \iff w_i>0) &=\frac{1}{\gd_i^2}\paren{1 - \sum_{k=1}^q \theta_k \gb_i^k}_+ \paren{ \sum_{i:x_i>0} \frac{1}{\gd_i^2}\paren{1 - \sum_{k=1}^q \theta_k \gb_i^k}}^{-1}
\end{align}
Note that the lagrange multiplier $\gl$ (and $\gl_w$) drops out entirely from the calculation, so it suffices to simply solve the following equation:
\begin{align}
w_i^* = \frac{\left(1-\sum_{k=1}^q\theta_k \gb_i^k\right)_+}{\gd_i^2}
\end{align}
such that 
\begin{align}
x_i = \frac{w_i^*}{\sum_{j=1}^p w_j}
\end{align}
 Uniqueness follows from the sufficiency and necessity of the KKT conditions.  $\theta$ must be computed in order to solve this problem.  Note that $\theta$ is the vector of elements defined by \req{thetadef}.  
 Define
 \begin{align}
 f_i(\theta) = \sum_{k=1}^q \theta_k \gb_i^k,
 \end{align}
 i.e. the sum of the $i$\textsuperscript{th} elements multiplied by $\theta_k$ across all factors, or the $i$\textsuperscript{th} row of $B$ times $\theta$.
 We have
 \begin{align}
 \theta_k &= \frac{\gs_k^2 G_k(w)}{\gl_w}  = \frac{\gs_k^2 {\gb^k}^\T w}{\gl_w}\label{thetadef}\\
 & = \frac{\gs_k^2}{\gl_w} \sum_{i=1}^p \gb_i^k \frac{\gl_w}{\gd_i^2} \paren{1 - \sum_{m=1}^q \theta_m \gb_i^m}\\
 &= \gs_k^2 \sum_{i=1}^p \frac{\paren{\gb_i^k -\gb_i^k\sum_{m=1}^q \theta_m(\gb_i^m)}\inds{\{ f_i(\theta)<1 \}}}{\gd_i^2}\\
 &= \gs_k^2 \sum_{i=1}^p \frac{\gb_i^k}{\gd_i^2}\inds{\{ f_i(\theta)<1 \}} -\gs_k^2 \sum_{i=1}^p \gb_i^k\sum_{m=1}^q\frac{\theta_m \gb_i^m}{\gd_i^2}\inds{\{ f_i(\theta)<1 \}} \\
 &= \psi_k(\theta)
 \end{align}
Note that for each $k=1,2,\ldots,q$, $psi_k(\theta)$ defines the fixed-point mapping:
\begin{align*}
\psi_k &: \bbr^q \to \bbr \quad \text{such that}\\
\theta_k &= \psi_k(\theta) = \psi_k(\theta_1, \theta_2,\ldots,\theta_k,\ldots,\theta_q)
\end{align*}
Solving for this fixed-point is equivalent to solving the KKT conditions; any solution to this fixed point will be a solution to the KKT system and any solution to the KKT system will solve this fixed point.  Therefore, the sufficiency of the KKT conditions for a unique minimizer, given the convexity of our constraints, guarantees that a fixed-point is the minimizer of this system.
 This minimizer can be computed approximately with a fixed-point iteration algorithm.  We define the following vectorized fixed-point equation:
 \begin{align}
\theta = \Psi(\theta) = [ \psi_1(\theta), \ldots, \psi_q(\theta)]^\T
 \end{align}
 and we wish to solve for $\theta$.
 In Matrix notation, we find that \req{thetadef} vectorized for $k=1,\ldots,q$ can be written as:
 \begin{align}
 \theta &= \frac{1}{\gl_w}\bV G(w) = \frac{1}{\gl_w} \mat{VB}^\T w\\
 &= \bV \paren{\mat{B}^\T \Gd^{-1}}\paren{\mat{\chi}_{\{\be-\mat{B} \theta>0\}}} - \bV \mat{B}^\T \Gd^{-1} \diag\paren{\mat{\chi}_{\{\be-\mat{B} \theta>0\}}}\mat{B} \theta
 \end{align}
 Multiplying  both sides by $\bV^{-1}$ ($\bV$ is full-rank, square, and diagonal) gives us:
 \begin{align}
\bV^{-1} \theta &=  \paren{\mat{B}^\T \Gd^{-1}}\paren{\mat{\chi}_{\{\be-\mat{B} \theta>0\}}} -  \mat{B}^\T \Gd^{-1} \diag\paren{\mat{\chi}_{\{\be-\mat{B} \theta>0\}}}\mat{B} \theta
\end{align} which, when we move the elements around, gives us:
\begin{align}
\paren{\bV^{-1} +\mat{B}^\T \Gd^{-1} \diag\paren{\mat{\chi}_{\{\be-\mat{B} \theta>0\}}}\mat{B}} \theta &= \paren{\mat{B}^\T \Gd^{-1}}\paren{\mat{\chi}_{\{\be-\mat{B} \theta>0\}}}
 \end{align}
Define:
\begin{align} \label{linsys}
\mat{c_\vartheta} = \Gd^{-1} \mat{\chi}_{\{\be-\mat{B} \vartheta>0\}}, \quad \mat{b_\vartheta} =\mat{B}^\T \mat{c_\vartheta}, \quad \mat{A_\vartheta} = \mat{V}^{-1} + \mat{B^\T \textbf{diag}(c_\vartheta) B}
\end{align}
Note that $\theta=[\theta_1, \ldots, \theta_q]$ solves:
\begin{align}
\mat{A}_\theta \theta = \mat{b}_\theta
\end{align}
We can solve for $\theta$ via the following fixed-point iteration algorithm:
\newcommand{\told}{\theta_{\text{old}}}
\newcommand{\tnew}{\theta_{\text{new}}}
\begin{algorithm}[hpt!]
\vspace{0.16in}
\begin{enumerate}[label=\textbf{\arabic*.},
  itemindent=0.32in, itemsep=0.032in]
\item Initialize $\told \leftarrow (0, \dots, 0) \in \bbR^q$ 
and tolerance $\epsilon > 0$.
\item Assemble $\mat{A}_{\told}$ and $b_{\told}$ 
from 
$(\bB, \bV, \bdel)$ and $\told$ as in $\req{linsys}$. \label{loop}
\item Compute $\tnew$ by solving 
$\mat{A}_{\told} \tnew = b_{\told}$. 
\item If $| \tnew - \told | >\epsilon$, 
update $\told \leftarrow \tnew$ and go to Step \ref{loop}
\item Compute $w=\bdel^{-1}(\rme-\mat{B} \tnew)_+$
and {\bf return} $\frac{w}{\be^\top w}$.
\end{enumerate}
\caption{(FFP). Given $(\mat{B}$, $\mat{V}$, $\bdel)$, computes
the optimizer of problem \req{con}. }
\label{alg}
\end{algorithm}
We refer to Algorithm \ref{alg} as the \emph{factor fixed point}
(FFP) method. 
\end{proof}
\newpage
\begin{example}
We spell out an explicit example with $q=2$ to show a more straightforward calculation.
Let 
\begin{align}
\bB &= [ \gb_1, \, \gb_2], \; \bV = \diag(\gs_1^2, \gs_2^2)\; \bdel =\diag(\gd_1^2 ,\ldots,\gd_p^2) \\
\bQ &= \bB \bV \bB^\T + \bdel
\end{align}
For a vector $u \in \bbr^p$, let $G_1(u) = \gb_1^\T u$ and the same for $G_2(u)$.  We therefore have our un-normalized solution for element $k$, $w_k$, with lagrange multiplier $\gl_w$:
\begin{align}
w_i &= \frac{1}{\gd_i^2}\paren{ \gl_w - \gs_1^2 G_1(w) \gb_i^1 - \gs_2^2 G_2(w) \gb_i^2}_+ \\
&= \frac{\gl_w}{\gd_i^2}\paren{1 -  \theta_1 \gb_i^1 - \theta_2 \gb_i^2}_+\\
w_i^* &= \frac{1}{\gd_i^2}\paren{1 -  \theta_1 \gb_i^1 - \theta_2 \gb_i^2}_+
\end{align}
In order to solve for $\theta$ we construct the following equations.  Let 
\begin{align*}
a_{1}(\theta) &= \frac{1}{\gs_1^2} + \sum_{i=1}^p \frac{\paren{\gb_i^1}^2}{\gd_i^2}\inds{\{\theta_1 \gb_i^1 + \theta_2 \gb_i^2<1\}} &\\
b_1(\theta) & = \sum_{i=1}^p \frac{\gb_i^1}{\gd_i^2}\inds{\{\theta_1 \gb_i^1 + \theta_2 \gb_i^2<1\}} \quad & c_{12}(\theta) = c_{21}(\theta) = \sum_{i=1}^p \frac{\gb_i^1 \gb_i^2}{\gd_i^2}\inds{\{\theta_1 \gb_i^1 + \theta_2 \gb_i^2<1\}}
\end{align*}
with $a_2(\theta)$ and $b_2(\theta)$ defined similarly.  We therefore have solve for $\theta$ by solving the following system:
\begin{align}\underbrace{\begin{pmatrix}
a_1(\theta) & c_{12}(\theta) \\
c_{21}(\theta) & a_2(\theta) 
\end{pmatrix}}_{\mat{A}_\theta}
\begin{pmatrix}
\theta_1 \\ \theta_2
\end{pmatrix}
= \underbrace{\begin{pmatrix}
b_1(\theta) \\ b_2(\theta)
\end{pmatrix}}_{\mat{b}_\theta}
\end{align}
Therefore, our fixed point equation is
\begin{align}
\Psi(\theta) = \mat{A}_\theta^{-1} \mat{b_\theta} = \theta
\end{align}
\end{example}


\end{document}






