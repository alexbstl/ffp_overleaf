\documentclass[11pt,leqno]{beamer}
\usepackage{makecell}
\usepackage{ifxetex}
\usepackage{wrapfig}
\usepackage{xcolor}
\ifxetex

  \usepackage[T1]{fontenc}
  \usepackage[no-math]{fontspec}

  \usefonttheme{serif}
  \setmainfont{Times NR MT Std}

  \usepackage[cal=euler, calscaled=1.0,
	      scr=boondox, scrscaled=1.0]{mathalfa}

  \let\hbar\relax
  \usepackage[noamssymbols,mtpcal,mtphrb,mtpfrak,
  	      subscriptcorrection,zswash]{mtpro2}
\else
  %\usepackage{lmodern}
  %\usepackage{mathrsfs}
  \usefonttheme{professionalfonts}
  \usepackage{stix}
  \newcommand{\mbf}{\mathbf} % to match mt2pro
\fi

\usetheme{metropolis}
\usepackage{appendixnumberbeamer}

%\usepackage{booktabs}
%\usepackage[scale=2]{ccicons}

%\usepackage{pgfplots}
%\usepgfplotslibrary{dateplot}

%\usepackage{xspace}
%\newcommand{\themename}{\textbf{\textsc{metropolis}}\xspace}

%\setbeamertemplate{bibliography item}{}


% BASIC PACKAGES %

\usepackage{makecell}
\usepackage{graphicx}
\usepackage{enumitem}
\usepackage{booktabs}
\usepackage{framed}
\usepackage{multirow}
\usepackage{scrextend}
\usepackage{stackengine}
\usepackage{subcaption}
\usepackage{comment}
\usepackage{url}
\usepackage{bm}
\usepackage[bottom]{footmisc}

% MARGIN NOTES %

\usepackage{marginnote}
\usepackage{setspace}

\newcommand{\lmn}[1]{{\reversemarginpar \setstretch{0.64} %
	\marginnote{{\scriptsize \bambootext{\emph{#1}}}}} }

\newcommand{\rmn}[1]{{\normalmarginpar \setstretch{0.64} %
	\marginnote{{\scriptsize \bambootext{\emph{#1}}}}} }

\setlength\marginparwidth{64pt}

% RAISE/LOWER %

\makeatletter
\newcommand{\raisemath}[1]{\mathpalette{\raisem@th{#1}}}
\newcommand{\raisem@th}[3]{\raisebox{#1}{$#2#3$}}
\makeatother

% HIDE MACRO %

\newif\ifhide
\hidetrue %\hidefalse
\newcommand{\hide}[1]{\ifhide {} \else {#1} \fi} 

% EMPHASIZE BOX %

\usepackage{empheq}
\newcommand{\widefbox}[1]{\fbox{\hspace{0.33in}#1\hspace{0.33in}}}


% CHAR MAPS: e.g. \calS for \mathcal{S}  %

\usepackage{pgffor}

\foreach \x in {A,...,Z}{%
  \expandafter\xdef\csname cal\x\endcsname{\noexpand %
	\ensuremath{\noexpand\mathcal{\x}}}
  \expandafter\xdef\csname scr\x\endcsname{\noexpand %
	\ensuremath{\noexpand\mathscr{\x}}}
  \expandafter\xdef\csname bb\x\endcsname{\noexpand %
	\ensuremath{\noexpand\mathbb{\x}}}
  \expandafter\xdef\csname rm\x\endcsname{\noexpand %
	\ensuremath{\noexpand\mathrm{\x}}}
  \expandafter\xdef\csname bf\x\endcsname{\noexpand %
	\ensuremath{\noexpand\mathbf{\x}}}
}

% ACCENTS %

\newcommand{\wh}[1]{\widehat{#1}}
\newcommand{\wt}[1]{\widetilde{#1}}

% SPACING %
\newcommand{\hs}[1]{ \hspace{#1pt} }

\newcommand{\mptt}{\hspace{-2pt}  }
\newcommand{\mpt}{ \hspace{-1pt}  }
\newcommand{\mhpt}{\hspace{-0.5pt}}
\newcommand{\hpt}{ \hspace{0.5pt} }

\newcommand{\pt}{ \hspace{1pt}  }
\newcommand{\ptt}{ \hspace{2pt} }
\newcommand{\pttt}{ \hspace{3pt}}
\newcommand{\ptttt}{\hspace{4pt}}

% BASIC MATH %

%\newcommand{\bst}{ \hspace{1.5pt} | \hspace{1.5pt} }
%\newcommand{\cst}{ \hspace{0.5pt} : \hspace{0.5pt} }
%\newcommand{\sst}{ \hspace{2pt} ; \hspace{0.5pt} }
\newcommand{\evl}{ \left| \right. }

\newcommand{\abs}[1]{\left| {#1} \right|}
\newcommand{\ceil}[1]{\left\lceil #1 \right\rceil}
\newcommand{\floor}[1]{\left\lfloor #1 \right\rfloor}

% ARROWS %

%\newcommand{\upto}{ \uparrow }
%\newcommand{\downto}{ \downarrow }
%\newcommand{\nto}{ \nrightarrow }

% TEXT %

%\newcommand{\tq}[1]{{\textquotedblleft #1\textquotedblright}}
%\newcommand{\ts}[2]{{#1\textsubscript{#2}}}


% Brackets, Parentheses, etc %

\newcommand{\pa}[1]{\hspace{1pt} \left( \hspace{0.5pt} %
	{#1} \hspace{0.5pt} \right) \hspace{1pt}}
\newcommand{\qa}[1]{\hspace{1pt} \left[ \hspace{0.5pt} %
	{#1} \hspace{0.5pt} \right] \hspace{1pt}}
\newcommand{\fa}[1]{\hspace{1pt} \left \{ \hspace{0.5pt} % 
	{#1} \hspace{0.5pt} \right \}\hspace{1pt}}
\newcommand{\za}[1]{\hspace{1pt} \left\langle \hspace{0.5pt} %
	{#1} \hspace{0.5pt} \right\rangle \hspace{1pt}}

\newcommand{\p}[1]{ \hspace{1pt} ( \hspace{0.25pt} %
	{#1} \hspace{0.25pt} ) \hspace{1pt}}
\newcommand{\q}[1]{ \hspace{1pt} [ \hspace{0.25pt} %
	{#1} \hspace{0.25pt} ] \hspace{1pt}}
\newcommand{\f}[1]{ \hspace{1pt} \{ \hspace{0.25pt} %
	{#1} \hspace{0.25pt} \} \hspace{1pt}}
\newcommand{\z}[1]{ \hspace{1pt} \langle \hspace{0.25pt} %
	{#1} \hspace{0.25pt}\rangle\hspace{1pt}}
\newcommand{\g}[1]{ \hspace{1pt} \langle \hspace{0.25pt} %
	{#1} \hspace{0.25pt}\rangle\hspace{1pt}}

\newcommand{\pp}[1]{ \hspace{1pt} \big( \hspace{0.5pt} %
	{#1} \hspace{0.5pt} \big) \hspace{1pt}}
\newcommand{\qq}[1]{ \hspace{1pt} \big[ \hspace{0.5pt} %
	{#1} \hspace{0.5pt} \big] \hspace{1pt}}
\newcommand{\ff}[1]{ \hspace{1pt} \big\{ \hspace{0.5pt} %
	{#1} \hspace{0.5pt} \big\} \hspace{1pt}}
\newcommand{\zz}[1]{ \hspace{1pt} \big\langle \hspace{0.5pt} %
	{#1} \hspace{0.5pt} \big\rangle \hspace{1pt}}

\newcommand{\ppp}[1]{ \hspace{1pt} \Big( \hspace{0.5pt} %
	{#1} \hspace{0.5pt} \Big) \hspace{1pt}}
\newcommand{\qqq}[1]{ \hspace{1pt} \Big[ \hspace{0.5pt} %
	{#1}  \hspace{0.5pt} \Big] \hspace{1pt}}
\newcommand{\fff}[1]{ \hspace{1pt} \Big\{ \hspace{0.5pt} %
	{#1}  \hspace{0.5pt} \Big\} \hspace{1pt}}
\newcommand{\zzz}[1]{ \hspace{1pt} \Big\langle \hspace{0.5pt} %
	{#1} \hspace{0.5pt} \Big\rangle \hspace{1pt}}

\newcommand{\pppp}[1]{ \hspace{1pt} \bigg(% 
	\hspace{0.5pt} {#1} \hspace{0.5pt} \bigg) \hspace{1pt}}
\newcommand{\qqqq}[1]{ \hspace{1pt} \bigg[% 
	\hspace{0.5pt} {#1} \hspace{0.5pt} \bigg] \hspace{1pt}}
\newcommand{\ffff}[1]{ \hspace{1pt} \bigg\{% 
	\hspace{0.5pt} {#1} \hspace{0.5pt} \bigg\} \hspace{1pt}}
\newcommand{\zzzz}[1]{ \hspace{1pt} \bigg\langle 
	\hspace{0.5pt} {#1} \hspace{0.5pt} \bigg\rangle \hspace{1pt}}


% REFERENCES & LABELS %

\makeatletter
\@ifpackageloaded{harvard}{}
  {\newcommand{\citeasnoun}{\cite}}
\makeatother

\newcommand{\ci}{\citeasnoun}
\newcommand{\cic}[2]{\citeasnoun[#1]{#2}}

\newcommand{\hci}[2]{ \href{#2}{\ci{#1}} }
\newcommand{\hcic}[3]{ \href{#3}{\cic{#1}{#2}} }
\newcommand{\hcite}[2]{ \href{#2}{\cite{#1}} }

\newcommand{\req}[1]{(\ref{#1})}






% PROBABILITY %

\newcommand{\cid}{\overset{\scrD}{\rightarrow}}
\newcommand{\eqd}{\overset{\scrD}{=}}
\newcommand{\cip}{\overset{p}{\rightarrow}}
\newcommand{\cas}{\overset{a.s.}{\rightarrow}}

\newcommand{\inds}[1]{ \mathbold{1}_{ \hspace{-1pt} {#1} }}
\newcommand{\indf}[1]{ \mathbold{1}_{ \hspace{-1pt} \f{ {#1} } }}
\newcommand{\var}{ \text{var} }
\newcommand{\cov}{ \text{cov} }

\newcommand{\Var}{ \mbf{Var}  }
\newcommand{\Cov}{ \mbf{Cov}  }
\newcommand{\Exp}{ \mbf{E}  }
\newcommand{\Prb}{ \mbf{P}  }
\newcommand{\Qrb}{ \mbf{Q} }

\newcommand{\re}{ \calE_a }

\renewcommand{\Re}{\mathfrak{R}}
\renewcommand{\Im}{\mathfrak{I}}


\newcommand{\rcll}{c{\`a}dl{\`a}g}


\newcommand{\blam}{ \bm{\Lambda} }




% Bamboo pallete %

\definecolor{deluge}{RGB}{124, 113, 173}
\definecolor{bamboo}{RGB}{220, 92, 5}
\definecolor{yellow}{RGB}{255, 172, 0}
\definecolor{orange}{RGB}{255, 144, 0}
\definecolor{oyster}{RGB}{151, 139, 125}
\definecolor{coral}{RGB}{199, 186, 167}
\definecolor{downy}{RGB}{110, 197, 184}
\definecolor{blueberry}{HTML}{6B7A8F}
\definecolor{alexbcolor}{HTML}{6088C7}

\def\delugetext#1{{\color{deluge}{{#1}}\color{deluge}}}
\def\bambootext#1{{\color{bamboo}{{#1}}\color{bamboo}}}
\def\yellowtext#1{{\color{yellow}{{#1}}\color{yellow}}}
\def\orangetext#1{{\color{orange}{{#1}}\color{orange}}}
\def\oystertext#1{{\color{oyster}{{#1}}\color{oyster}}}
\def\coraltext#1{{\color{coral}{{#1}}\color{coral}}}
\def\downytext#1{{\color{downy}{{#1}}\color{downy}}}



\setbeamercolor{frametitle}{bg=blueberry}
\setbeamercolor{progress bar}{fg=bamboo, bg=oyster}
\setbeamercolor{alerted text}{fg=bamboo}



\title{Plan for Summer 2020}
\date{\today} 
\author{\texorpdfstring{Alex Bernstein\\}
\texorpdfstring{\url{abernstein@ucsb.edu}\\}
{\it (joint work with Alex Shkolnik)}} 
\institute{
{\it Department of Statistics \& Applied Probability\\ }
University of California, Santa Barbara
% \titlegraphic{\hfill\includegraphics[height=1.5cm]{logo.pdf}}
} 

\begin{document}


\ifxetex
  \let\lsum\sum
  \renewcommand{\sum}{\bm{\lsum}}

  \let\lprod\prod
  \renewcommand{\prod}{\bm{\lprod}}
\else
\fi




\begin{frame}
\maketitle
\end{frame}

\newcommand{\af}{Y}
\newcommand{\df}{X}

\begin{frame}{Part I: FFP and Extensions}
\begin{enumerate}[a)]
\item Elucidate Applications for FFP in the Current Context:
\begin{itemize}
\item Sensitivity Analysis (These derivatives are written in terms of the one-factor case, but should generalize):

\begin{itemize}
\item Weights vs. Asset Loadings: $\diff{\gb_j} w_i$ in both cases where $i \neq j$ and $i=j$
\item Weights vs. Factor Variance: $\diff{\gs^2} w_i$ 
\item Weights vs. Specific Variance: $\diff{\gd_i^2} w_i$ (0 if $i \neq j$)
\item Fixed Point vs. Asset Loadings: $\diff{\gb_i}\theta= \diff{\gb_i}{\psi(\theta)}$ at FP
\item Fixed Point vs. Factor Variance: $\diff{\gs^2}\theta= \diff{\gs^2}{\psi(\theta)}$ at FP
\item Fixed Point vs. Specific Variance: $\diff{\gd_i^2}\theta= \diff{\gd_i^2}{\psi(\theta)}$ at FP
\end{itemize}
\item Applications to other restricted QP problems
\end{itemize}
\item Complete proof for convergence of FFP algorithm for $q>1$ factors
\item See if any progress can be made on linear objective case and/or generalizing of constraints
\end{enumerate}
\end{frame}


\begin{frame}{Lead}
Typically, no explicit formula exists for a minimum-variance portfolio, which are computed via the well-developed theory of numerical convex optimization. Under a factor-covariance model, we develop a semi-explicit formula and methodology for Minimum-Variance (Markowitz) Portfolios.
\end{frame}

\begin{frame}{Objective(Funnel)-Challenge-Action-Resolution}
\begin{itemize}
\item Opening
\begin{itemize}
\item Minimum Variance portfolios and convex optimization is well developed and studied
\item Well-understood computational theory and algorithms; less well-understood is when explicit formulae are available and sensitivity analysis can be conducted in a (relatively) computationally efficient manner
\end{itemize}
\item Funnel
\begin{itemize}
\item We are working in the context of minimum variance portfolios; this generalizes to other element wise-constrained quadratic programs without a linear constraint/term
\end{itemize}
\item Challenge
\begin{itemize}
\item We aim to show that a semi-explicit formula for minimum-variance portfolios exist for a single factor, (and can generalize to any arbitrary number of factors), and can be computed very quickly
\end{itemize}
\end{itemize}
\end{frame}

\begin{frame}{OCAR Part II}
\begin{itemize}
\item Action
\begin{itemize}
\item We derive a fixed-point equation, the solution of which can be used to construct the minimum-variance portfolio.  
\item We show that a fixed-point iteration algorithm can be used to compute the necessary fixed point, with guaranteed convergence
\end{itemize}
\item Resolution-\\
We are able to expand our understanding of elementwise-constrained (e.g. long-only) minimum-variance portfolios, as our formula allows us to compute partial derivatives and conduct a sensitivity analysis.
\end{itemize}
\end{frame}

\begin{frame}{Part II: Random Matrix Theory Extensions}
\begin{enumerate}
\item Write a good background on classical RMT results
\begin{itemize}
\item What is ``convergence'' in the RMT sense?
\item Different regimes for convergence- CLT; Marcenko-Pastur
\item What does the matrix eigenstructure tell us?
\end{itemize}
\item Applications of spiked covariance results to eigenstructure of factor models 
\end{enumerate}
\end{frame}

\begin{frame}{To Do's and Dates:}
Part I:
\begin{itemize}
\item Clean up FFP proof- Due 6/27
\item Compute listed Partial Derivatives- Due 6/27
\item Work out two-factor example for generalization of proof- Due 6/29
\end{itemize}
Part II:
\begin{itemize}
\item RMT Background: Classical Results- Due 7/1
\item Spiked Covariance Results- Due 7/6
\item Simulations- Due 7/10
\end{itemize}
\end{frame}


\end{document}

