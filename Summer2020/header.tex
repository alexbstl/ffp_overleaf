

% BASIC PACKAGES %

\usepackage{harvard}
\usepackage{graphicx}
\usepackage{enumerate}
\usepackage{url}
\usepackage{framed}
\usepackage{float}
\usepackage{comment}
\usepackage{multirow}
\usepackage{makecell}
\usepackage{stackengine}
\usepackage{bm}

% MARGIN NOTES %

\usepackage{marginnote}
\usepackage{setspace}

\newcommand{\mnl}[1]{{\reversemarginpar \setstretch{0.64} %
	\marginnote{{\scriptsize \bluetext{\emph{#1}}}}} }

\newcommand{\mnr}[1]{{\normalmarginpar \setstretch{0.64} %
	\marginnote{{\scriptsize \bluetext{\emph{#1}}}}} }

\setlength\marginparwidth{64pt}

% ALGORITHMS %

\floatstyle{ruled}
\newfloat{algorithm}{hpt}{lop}
\floatname{algorithm}{Algorithm}


% TABLE PACKAGES %
\usepackage{booktabs}
\usepackage{xcolor}

% RAISE/LOWER %

\makeatletter
\newcommand{\raisemath}[1]{\mathpalette{\raisem@th{#1}}}
\newcommand{\raisem@th}[3]{\raisebox{#1}{$#2#3$}}
\makeatother

% HIDE MACRO %

\newif\ifhide
\hidetrue %\hidefalse
\newcommand{\hide}[1]{\ifhide {} \else {#1} \fi} 

% EMPHASIZE BOX %

\usepackage{empheq}
\newcommand{\widefbox}[1]{\fbox{\hspace{0.33in}#1\hspace{0.33in}}}



% ACCENTS %

\newcommand{\wh}[1]{\widehat{#1}}
\newcommand{\wt}[1]{\widetilde{#1}}



% PROOFS %

\usepackage{amsthm}

\newtheoremstyle{mytheoremstyle}
{\topsep}                    % Space above
{\topsep}                    % Space below
{\itshape}                   % Body font
{}                           % Indent amount
{\bfseries}                  % Theorem head font
{.}                          % Punctuation after theorem head
{.5em}                       % Space after theorem head
{}  % Theorem head spec (can be left empty, meaning ‘normal’)

\theoremstyle{mytheoremstyle}
\newtheorem{thm}[theorem]{Theorem}

\newtheorem*{theo}{Theorem}
\newtheorem*{lemm}{Lemma}
\newtheorem*{defn}{Definition}
\newtheorem*{prop}{Proposition}
\newtheorem*{assm}{Assumption}
\newtheorem*{cond}{Condition}
\newtheorem*{remk}{Remark}
\newtheorem*{corr}{Corollary}


% REFERENCES & LABELS %

\usepackage{hyperref}
\hypersetup{colorlinks=true, allcolors=bamboo}


\newcommand{\ci}[1]{\citeasnoun{#1}}
\newcommand{\cic}[2]{\citeasnoun[#1]{#2}}

\newcommand{\hci}[2]{ \href{#2}{\ci{#1}} }
\newcommand{\hcic}[3]{ \href{#3}{\cic{#1}{#2}} }
\newcommand{\hcite}[2]{ \href{#2}{\cite{#1}} }

\newcommand{\req}[1]{(\ref{#1})}




