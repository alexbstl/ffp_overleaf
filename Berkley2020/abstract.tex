\documentclass{article}
\usepackage{titling}
\usepackage{blindtext}
\title{A Closed-Form Solution to the Markowitz Portfolio Problem}
\author{Alex Bernstein and Alex Shkolnik}
\begin{document}
\begin{titlingpage}
    \maketitle
    \begin{abstract}
      In 1952, Harry Markowitz transformed finance by framing the portfolio construction problem as a tradeoff between the mean and the variance of return. This application of quadratic optimization is at the basis of breakthroughs such as the Capital Asset Pricing Model (CAPM) and Arbitrage Pricing Theory (APT). The classical Markowitz problem may be solved in closed form. However, when the portfolio weights face inequality constraints, one has to resort to a numerical optimization routine. This occurs for constrains as simple and as useful in practice as the long only constraint. We show that it is still possible to obtain closed form solution to this and related constrained problems when we assume a factor model. This approach provides significant gains in either accuracy or computational efficiency. Through our closed-form formulae we are also able to study the structure of the composition and the sensitivity of constrained Markowitz portfolios in terms of various macroeconomic variables. We illustrate our results will several case studies.
    \end{abstract}
\end{titlingpage}
\end{document}