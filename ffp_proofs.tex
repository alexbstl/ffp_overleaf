\documentclass[12pt,leqno,letterpaper]{article}
\usepackage{ulem}

% PAGE SETTINGS %

\parskip 0in
\parindent 0.32in
%\voffset -0.125in

\usepackage[letterpaper]{geometry}
\geometry{text={5.562306in,9in}, centering} % golden ratio :)
%\renewcommand{\baselinestretch}{1.02} 

% FONTS %

\usepackage{ifxetex}
\ifxetex
  \usepackage[T1]{fontenc}
  \usepackage[no-math]{fontspec}

  \setmainfont[Scale=1.0]{Times NR MT Std}

  \usepackage[cal=euler, calscaled=1,
	      scr=boondox, scrscaled=1]{mathalfa}

  \usepackage[mtpcal,mtphrb,mtpfrak,noamssymbols,nofontinfo,
  	      subscriptcorrection,zswash]{mtpro2}

  %\DeclareMathSizes{12}{13}{9}{7}
\else
  \usepackage{lmodern}
  \usepackage{amsfonts}
  \usepackage{amssymb}
  \usepackage{mathrsfs}
  % to match mt2pro
  \newcommand{\mbf}{\mathbf} 
  \newcommand{\what}[1]{\widehat{#1}}
\fi

% THEOREMS %

\usepackage{amsthm}

\newtheoremstyle{mytheoremstyle}
{\topsep}                    % Space above
{\topsep}                    % Space below
{\itshape}                   % Body font
{}                           % Indent amount
{\bfseries}                  % Theorem head font
{.}                          % Punctuation after theorem head
{.5em}                       % Space after theorem head
{}  % Theorem head spec (can be left empty, meaning ‘normal’)

\theoremstyle{mytheoremstyle}

\makeatletter
\@ifclassloaded{report}{
   \newtheorem{theorem}{Theorem} }
{  \newtheorem{theorem}{Theorem}[section] }
\makeatother

\newtheorem{thm}[theorem]{Theorem}
\newtheorem{proposition}[theorem]{Proposition}
\newtheorem{lemma}[theorem]{Lemma}
\newtheorem{corollary}[theorem]{Corollary}
\newtheorem{definition}[theorem]{Definition}
\newtheorem{remark}[theorem]{Remark}
\newtheorem{example}[theorem]{Example}
\newtheorem{condition}[theorem]{Condition}
\newtheorem{assumption}[theorem]{Assumption}

\newtheorem*{theo}{Theorem}
\newtheorem*{lemm}{Lemma}
\newtheorem*{defn}{Definition}
\newtheorem*{prop}{Proposition}
\newtheorem*{assm}{Assumption}
\newtheorem*{cond}{Condition}
\newtheorem*{remk}{Remark}
\newtheorem*{expl}{Example}


% REFERENCES %

\usepackage{harvard}

% SECTIONS %

\usepackage{titlesec}

\titleformat{\section}[runin]
{\normalfont\bfseries}
{\thesection.}{0.08in}{}[.]

\titlespacing{\section}{\parindent}{0.08in}{0.08in}

\titleformat{\subsection}[runin]
{\normalfont\slshape}
{\thesubsection.}{0.04in}{}[.]

\titlespacing{\subsection}{\parindent}{0.08in}{0.08in}


% ABSTRACT

\renewcommand{\abstractname}{\vspace{-0.48in}}

\renewenvironment{abstract}
{\small
\begin{center}
  \bfseries \abstractname\vspace{-.5em}\vspace{0pt}
\end{center}
\list{}{
	\setlength{\leftmargin}{\parindent}
    \setlength{\rightmargin}{\leftmargin}
    \setlength{\parsep}{0in}
	\itemindent \parindent} 
  \item\relax}
{\endlist}

% TABLES AND FIGURES


\usepackage{caption}
\captionsetup[figure]{margin=0.32in,
	labelsep=period,labelfont=bf,textfont=it}
\captionsetup[table]{margin=0.32in,
	labelsep=period,labelfont=bf,textfont=it}





% BASIC PACKAGES %

\usepackage{makecell}
\usepackage{graphicx}
\usepackage{enumitem}
\usepackage{booktabs}
\usepackage{framed}
\usepackage{multirow}
\usepackage{scrextend}
\usepackage{stackengine}
\usepackage{subcaption}
\usepackage{comment}
\usepackage{url}
\usepackage{bm}
\usepackage[bottom]{footmisc}

% MARGIN NOTES %

\usepackage{marginnote}
\usepackage{setspace}

\newcommand{\lmn}[1]{{\reversemarginpar \setstretch{0.64} %
	\marginnote{{\scriptsize \bambootext{\emph{#1}}}}} }

\newcommand{\rmn}[1]{{\normalmarginpar \setstretch{0.64} %
	\marginnote{{\scriptsize \bambootext{\emph{#1}}}}} }

\setlength\marginparwidth{64pt}

% RAISE/LOWER %

\makeatletter
\newcommand{\raisemath}[1]{\mathpalette{\raisem@th{#1}}}
\newcommand{\raisem@th}[3]{\raisebox{#1}{$#2#3$}}
\makeatother

% HIDE MACRO %

\newif\ifhide
\hidetrue %\hidefalse
\newcommand{\hide}[1]{\ifhide {} \else {#1} \fi} 

% EMPHASIZE BOX %

\usepackage{empheq}
\newcommand{\widefbox}[1]{\fbox{\hspace{0.33in}#1\hspace{0.33in}}}


% CHAR MAPS: e.g. \calS for \mathcal{S}  %

\usepackage{pgffor}

\foreach \x in {A,...,Z}{%
  \expandafter\xdef\csname cal\x\endcsname{\noexpand %
	\ensuremath{\noexpand\mathcal{\x}}}
  \expandafter\xdef\csname scr\x\endcsname{\noexpand %
	\ensuremath{\noexpand\mathscr{\x}}}
  \expandafter\xdef\csname bb\x\endcsname{\noexpand %
	\ensuremath{\noexpand\mathbb{\x}}}
  \expandafter\xdef\csname rm\x\endcsname{\noexpand %
	\ensuremath{\noexpand\mathrm{\x}}}
  \expandafter\xdef\csname bf\x\endcsname{\noexpand %
	\ensuremath{\noexpand\mathbf{\x}}}
}

% ACCENTS %

\newcommand{\wh}[1]{\widehat{#1}}
\newcommand{\wt}[1]{\widetilde{#1}}

% SPACING %
\newcommand{\hs}[1]{ \hspace{#1pt} }

\newcommand{\mptt}{\hspace{-2pt}  }
\newcommand{\mpt}{ \hspace{-1pt}  }
\newcommand{\mhpt}{\hspace{-0.5pt}}
\newcommand{\hpt}{ \hspace{0.5pt} }

\newcommand{\pt}{ \hspace{1pt}  }
\newcommand{\ptt}{ \hspace{2pt} }
\newcommand{\pttt}{ \hspace{3pt}}
\newcommand{\ptttt}{\hspace{4pt}}

% BASIC MATH %

%\newcommand{\bst}{ \hspace{1.5pt} | \hspace{1.5pt} }
%\newcommand{\cst}{ \hspace{0.5pt} : \hspace{0.5pt} }
%\newcommand{\sst}{ \hspace{2pt} ; \hspace{0.5pt} }
\newcommand{\evl}{ \left| \right. }

\newcommand{\abs}[1]{\left| {#1} \right|}
\newcommand{\ceil}[1]{\left\lceil #1 \right\rceil}
\newcommand{\floor}[1]{\left\lfloor #1 \right\rfloor}

% ARROWS %

%\newcommand{\upto}{ \uparrow }
%\newcommand{\downto}{ \downarrow }
%\newcommand{\nto}{ \nrightarrow }

% TEXT %

%\newcommand{\tq}[1]{{\textquotedblleft #1\textquotedblright}}
%\newcommand{\ts}[2]{{#1\textsubscript{#2}}}


% Brackets, Parentheses, etc %

\newcommand{\pa}[1]{\hspace{1pt} \left( \hspace{0.5pt} %
	{#1} \hspace{0.5pt} \right) \hspace{1pt}}
\newcommand{\qa}[1]{\hspace{1pt} \left[ \hspace{0.5pt} %
	{#1} \hspace{0.5pt} \right] \hspace{1pt}}
\newcommand{\fa}[1]{\hspace{1pt} \left \{ \hspace{0.5pt} % 
	{#1} \hspace{0.5pt} \right \}\hspace{1pt}}
\newcommand{\za}[1]{\hspace{1pt} \left\langle \hspace{0.5pt} %
	{#1} \hspace{0.5pt} \right\rangle \hspace{1pt}}

\newcommand{\p}[1]{ \hspace{1pt} ( \hspace{0.25pt} %
	{#1} \hspace{0.25pt} ) \hspace{1pt}}
\newcommand{\q}[1]{ \hspace{1pt} [ \hspace{0.25pt} %
	{#1} \hspace{0.25pt} ] \hspace{1pt}}
\newcommand{\f}[1]{ \hspace{1pt} \{ \hspace{0.25pt} %
	{#1} \hspace{0.25pt} \} \hspace{1pt}}
\newcommand{\z}[1]{ \hspace{1pt} \langle \hspace{0.25pt} %
	{#1} \hspace{0.25pt}\rangle\hspace{1pt}}
\newcommand{\g}[1]{ \hspace{1pt} \langle \hspace{0.25pt} %
	{#1} \hspace{0.25pt}\rangle\hspace{1pt}}

\newcommand{\pp}[1]{ \hspace{1pt} \big( \hspace{0.5pt} %
	{#1} \hspace{0.5pt} \big) \hspace{1pt}}
\newcommand{\qq}[1]{ \hspace{1pt} \big[ \hspace{0.5pt} %
	{#1} \hspace{0.5pt} \big] \hspace{1pt}}
\newcommand{\ff}[1]{ \hspace{1pt} \big\{ \hspace{0.5pt} %
	{#1} \hspace{0.5pt} \big\} \hspace{1pt}}
\newcommand{\zz}[1]{ \hspace{1pt} \big\langle \hspace{0.5pt} %
	{#1} \hspace{0.5pt} \big\rangle \hspace{1pt}}

\newcommand{\ppp}[1]{ \hspace{1pt} \Big( \hspace{0.5pt} %
	{#1} \hspace{0.5pt} \Big) \hspace{1pt}}
\newcommand{\qqq}[1]{ \hspace{1pt} \Big[ \hspace{0.5pt} %
	{#1}  \hspace{0.5pt} \Big] \hspace{1pt}}
\newcommand{\fff}[1]{ \hspace{1pt} \Big\{ \hspace{0.5pt} %
	{#1}  \hspace{0.5pt} \Big\} \hspace{1pt}}
\newcommand{\zzz}[1]{ \hspace{1pt} \Big\langle \hspace{0.5pt} %
	{#1} \hspace{0.5pt} \Big\rangle \hspace{1pt}}

\newcommand{\pppp}[1]{ \hspace{1pt} \bigg(% 
	\hspace{0.5pt} {#1} \hspace{0.5pt} \bigg) \hspace{1pt}}
\newcommand{\qqqq}[1]{ \hspace{1pt} \bigg[% 
	\hspace{0.5pt} {#1} \hspace{0.5pt} \bigg] \hspace{1pt}}
\newcommand{\ffff}[1]{ \hspace{1pt} \bigg\{% 
	\hspace{0.5pt} {#1} \hspace{0.5pt} \bigg\} \hspace{1pt}}
\newcommand{\zzzz}[1]{ \hspace{1pt} \bigg\langle 
	\hspace{0.5pt} {#1} \hspace{0.5pt} \bigg\rangle \hspace{1pt}}


% REFERENCES & LABELS %

\makeatletter
\@ifpackageloaded{harvard}{}
  {\newcommand{\citeasnoun}{\cite}}
\makeatother

\newcommand{\ci}{\citeasnoun}
\newcommand{\cic}[2]{\citeasnoun[#1]{#2}}

\newcommand{\hci}[2]{ \href{#2}{\ci{#1}} }
\newcommand{\hcic}[3]{ \href{#3}{\cic{#1}{#2}} }
\newcommand{\hcite}[2]{ \href{#2}{\cite{#1}} }

\newcommand{\req}[1]{(\ref{#1})}






% BIBLIO %

\usepackage{harvard}
\usepackage{hyperref}
\hypersetup{colorlinks=true, allcolors=bamboo}

% TABLES %

\usepackage{booktabs}

% ALGORITHMS %

\usepackage{float}
\floatstyle{ruled}
\newfloat{algorithm}{hpt}{lop}
\floatname{algorithm}{Algorithm}

\usepackage{algorithm}
\usepackage{algpseudocode}










%\newif\ifhide
%\hidetrue %\hidefalse
%\newcommand{\hide}[1]{\ifhide {} \else {#1} \fi} 



% SPACING/TEXT %

\newcommand{\s}[1][1]{\hspace{#1pt}}
\newcommand{\mst}{{\scalebox{0.6}[0.6]{\(\hspace{-1pt}{*}\)}}}

\newcommand{\tq}[1]{{\textquotedblleft #1\textquotedblright}}
\newcommand{\ts}[2]{{#1\textsubscript{#2}}}


% PROBABILITY %

\newcommand{\cid}{\overset{\scrD}{\rightarrow}}
\newcommand{\eqd}{\overset{\scrD}{=}}
\newcommand{\cip}{\overset{p}{\rightarrow}}
\newcommand{\cas}{\overset{a.s.}{\rightarrow}}
\newcommand{\sip}{\overset{p}{\sim}}
\newcommand{\sas}{\overset{a.s.}{\sim}}

\newcommand{\setst}{\mathbb} % set style
\newcommand{\oprst}{\textsc} % operator style

\newcommand{\inds}[1]{ \mbf{1}_{ \s[-0.5] {#1} }}
\newcommand{\indf}[1]{ \mbf{1}_{ \s[-0.5] \{#1\} }}

\newcommand{\var}{\oprst{var}}
\newcommand{\std}{\oprst{std}}
\newcommand{\cov}{\oprst{cov}}
\newcommand{\Var}{\oprst{Var}}
\newcommand{\Cov}{\oprst{Cov}}
\newcommand{\Std}{\oprst{SD}}
\newcommand{\MSE}{\oprst{MSE}}
\newcommand{\RMSE}{\oprst{RMSE} }
\newcommand{\Exp}{\oprst{E}\s[.5]}
\newcommand{\Prb}{\oprst{P}\s[.5]}
\newcommand{\Qrb}{\ensuremath{\oprst{Q}\s[.5]}}

% SETS / MATH $

\newcommand{\nn}{\bbN}
\newcommand{\rn}{\bbR}
\newcommand{\zn}{\bbZ}
\newcommand{\qn}{\bbQ}

\newcommand{\Img}{\mathrm{Im}}
\newcommand{\Rea}{\mathrm{Re}}
\newcommand{\im}{\mathrm{i}}


\newcommand{\bst}{ \s[1.5] | \s[1.5] }
\newcommand{\cst}{ \s[0.5] : \s[0.5] }
\newcommand{\sst}{ \s[2.0] ; \s[0.5] }


\newcommand{\mat}[1]{\mbf{#1}}
\newcommand{\ones}{\rme}
\newcommand{\tr}{\textsc{Tr}}   


% GREEK %

\newcommand{\eps}{\epsilon}


% LETTER STYLES %
\usepackage{pgffor}

\foreach \x in {A,B,...,Z,a,b,...,z} 
{
  \expandafter\xdef\csname cal\x\endcsname{\noexpand 
	\ensuremath{\noexpand\mathcal{\x}}}
  \expandafter\xdef\csname scr\x\endcsname{\noexpand 
	\ensuremath{\noexpand\mathscr{\x}}}
  \expandafter\xdef\csname bb\x\endcsname{\noexpand 
	\ensuremath{\noexpand\mathbb{\x}}}
  \expandafter\xdef\csname rm\x\endcsname{\noexpand 
	\ensuremath{\noexpand\mathrm{\x}}}
  \expandafter\xdef\csname bf\x\endcsname{\noexpand 
	\ensuremath{\noexpand\mbf{\x}}}
}

% ARROWS %

\newcommand{\upto}{ \uparrow }
\newcommand{\downto}{ \downarrow }
\newcommand{\nto}{ \nrightarrow }



% COLABORATION %


\newcommand{\lrgtext}[1]{\oystertext{#1 -- lisa}}
\newcommand{\adstext}[1]{\bambootext{#1 -- alex}}
\newcommand{\agptext}[1]{\downytext{#1}}


\newcommand{\lrglmn}[1]{{\reversemarginpar \setstretch{0.64} %
	\marginnote{{\scriptsize 
	\oystertext{\emph{#1}--lisa}}}} }
\newcommand{\lrgrmn}[1]{{\normalmarginpar \setstretch{0.64} %
	\marginnote{{\scriptsize 
	\oystertext{\emph{#1}--lisa}}}} }

\newcommand{\adslmn}[1]{{\reversemarginpar \setstretch{0.64} %
	\marginnote{{\scriptsize 
	\bambootext{\emph{#1}--alex.s}}}} }
\newcommand{\adsrmn}[1]{{\normalmarginpar \setstretch{0.64} %
	\marginnote{{\scriptsize 
	\bambootext{\emph{#1} --alex.s}}}} }

\newcommand{\agplmn}[1]{{\reversemarginpar \setstretch{0.64} %
	\marginnote{{\scriptsize 
	\downytext{\emph{#1}--alex.p}}}}  }
\newcommand{\agprmn}[1]{{\normalmarginpar \setstretch{0.64} %
	\marginnote{{\scriptsize 
	\downytext{\emph{#1}--alex.p}}}}  }





% LOCAL %

\newcommand{\nv}{p}
\newcommand{\no}{n}
\newcommand{\tix}{j}
\newcommand{\six}{i}




\newcommand{\nb}{b}
\newcommand{\nh}{h}
\newcommand{\nz}{z}

\newcommand{\rrv}{Y}
\newcommand{\frv}{X}
\newcommand{\srv}{Z}

\newcommand{\ret}{\rmY}
\newcommand{\fet}{\rmX}
\newcommand{\set}{\rmZ}

\newcommand{\red}{\mbf{Y}}
\newcommand{\zed}{\mbf{Z}}

\newcommand{\bsig}{{\bm{\Sigma}}}
\newcommand{\bdel}{{\bm{\Delta}}}

\newcommand{\sphp}{\bbS^{\no-1}}
\newcommand{\covb}{{\bm{\Sigma}}}
\newcommand{\tfvol}{{\sigma}}
\newcommand{\tfv}{{\sigma^2}} 
\newcommand{\tsvol}{{\delta}}
\newcommand{\tsv}{{\delta^2}} 

\newcommand{\tfvolh}{\hat{\sigma}}
\newcommand{\tsvolh}{\hat{\delta}}
\newcommand{\tsvh}{{\hat{\delta}^2}} 
\newcommand{\tfvh}{{\hat{\sigma}^2}} 

\newcommand{\ter}{\scrT}
\newcommand{\vfr}{\scrR}
\newcommand{\vol}{\rmV}


\newcommand{\scov}{\mat{S}}
\newcommand{\covh}{\bm{\hat{\Sigma}}}
\newcommand{\efvol}{\ensuremath{\hat \sigma}}
\newcommand{\efv}{\ensuremath{\hat{\sigma}^2}} 
\newcommand{\esvol}{\ensuremath{\hat \delta}}
\newcommand{\esv}{\ensuremath{\hat{\delta}^2}} 
\newcommand{\esvi}{\ensuremath{\hat{\delta_i}^2}} 


\newcommand{\err}{\scrE}
\newcommand{\spar}{\tau}
\newcommand{\vpar}{t}
\newcommand{\covc}{\bm{\Sigma}_\spar}

%\newcommand{\minw}{w_{\scalebox{0.6}[0.6]{\(\hspace{-1pt}{h}\)}}}
%\newcommand{\optw}{w_{\scalebox{0.6}[0.6]{\(\hspace{-1pt}{b}\)}}}

\newcommand{\minw}{\hat{w}}
\newcommand{\optw}{w}
\newcommand{\eqww}{w_{\rme}}


\newcommand{\diag}{\textbf{diag}}    % factor exposure

\newcommand{\prp}[2]{\langle #1, #2 \rangle_{\hspace{-1pt}\nv}}
\newcommand{\prj}[2]{\langle #1, #2 \rangle}
\newcommand{\pri}[2]{\langle #1, #2 \rangle_\infty}
\newcommand{\ang}[2]{\theta_{\prj{#1}{#2}}}
\newcommand{\anp}[2]{\theta_{\prp{#1}{#2}}}
\newcommand{\Ang}[2]{\Theta_{\prj{#1}{#2}}}
\newcommand{\Anp}[2]{\Theta_{\prp{#1}{#2}}}


\newcommand{\chb}{\psi}
\newcommand{\eig}{\rms^2_{\hspace{-1pt} \nv}}
\newcommand{\blk}{\ell^2_{\hspace{-1pt} \nv}}
\newcommand{\sig}{\rms_\nv}

\newcommand{\me}{\mu}
\newcommand{\mei}{\mu_\infty}
\newcommand{\mep}{\mu_\nv}
\newcommand{\cv}{\rmd}
\newcommand{\cvp}{\rmd_\nv}
\newcommand{\cvi}{\rmd_\infty}



\newcommand{\llns}{\eta}
\newcommand{\svec}{\varphi}
\newcommand{\sy}{\chi_\nv}
\newcommand{\syk}{\chi_{\nv_k}}
\newcommand{\sign}{\sigma^2_{\rmX}}
\newcommand{\sigb}{\tfvol_{\hspace{-1.16pt} \nv}^2}
\newcommand{\sigh}{\tfvolh_{\hspace{-1.16pt} \nv}^2}

\newcommand{\thr}{\rho}
\newcommand{\roh}{\kappa}

%%%%% Commands Alex B %%%%%%%%%%


\newcommand{\ga}{\alpha}
\newcommand{\gb}{\beta}
\newcommand{\gc}{y}
\newcommand{\gd}{\delta}
\newcommand{\gf}{\phi}
\newcommand{\gl}{\lambda}
\newcommand{\gk}{\kappa}
\newcommand{\go}{\omega}
\newcommand{\gt}{\theta}
\newcommand{\gr}{\rho}
\newcommand{\gs}{\sigma}

\newcommand{\Gf}{\Phi}
\newcommand{\Go}{\Omega}
\newcommand{\Gc}{\Gamma}
\newcommand{\Gth}{\theta}
\newcommand{\Gd}{\Delta}
\newcommand{\Gs}{\Sigma}
\newcommand{\Gl}{\Lambda}
\renewcommand{\gc}{\gamma}
\newcommand{\mrm}{\mathrm}
\newcommand{\geps}{\varepsilon}

\newcommand{\T}{\top}

\newcommand{\paren}[1]{\left(#1\right)}

\newcommand{\ncal}{\mathcal{N}}


%Calculus
\newcommand{\diff}[1]{\frac{\partial }{ \partial {#1}}}


% Bamboo pallete %

\definecolor{deluge}{RGB}{124, 113, 173}
\definecolor{bamboo}{RGB}{220, 92, 5}
\definecolor{yellow}{RGB}{255, 172, 0}
\definecolor{orange}{RGB}{255, 144, 0}
\definecolor{oyster}{RGB}{151, 139, 125}
\definecolor{coral}{RGB}{199, 186, 167}
\definecolor{downy}{RGB}{110, 197, 184}
\definecolor{blueberry}{HTML}{6B7A8F}
\definecolor{alexbcolor}{HTML}{6088C7}

\def\delugetext#1{{\color{deluge}{{#1}}\color{deluge}}}
\def\bambootext#1{{\color{bamboo}{{#1}}\color{bamboo}}}
\def\yellowtext#1{{\color{yellow}{{#1}}\color{yellow}}}
\def\orangetext#1{{\color{orange}{{#1}}\color{orange}}}
\def\oystertext#1{{\color{oyster}{{#1}}\color{oyster}}}
\def\coraltext#1{{\color{coral}{{#1}}\color{coral}}}
\def\downytext#1{{\color{downy}{{#1}}\color{downy}}}






%\newcommand{\ind}[1]{{1}_{\{#1\}}}
\begin{document}

%\renewcommand{\labelenumi}{(\arabic{enumi})}
% Enable bold sum and product symbols
\ifxetex
  \let\lsum\sum
  \renewcommand{\sum}{\bm{\lsum}} 

  \let\lprod\prod
  \renewcommand{\prod}{\bm{\lprod}}
\else
\fi

\author{Alex Bernstein\footnote{Department of Statistics
\& Applied Probability, UCSB. Email:~{\tt abernstein@ucsb.edu}.} 
  ~~~ and ~~ Alex Shkolnik\footnote{Department of Statistics
\& Applied Probability, UCSB. Email:~{\tt shkolnik@ucsb.edu}.}}


\title{\vspace{-0.16in}{\bf \Large  Semi-explicit Solutions to 
Structured \\ Quadratic Programming}}

\date{\today}

%\newcommand{\thr}{\theta}
%\newcommand{\gfn}{G}
%\newcommand{\bB}{\mat{B}}
%\newcommand{\bQ}{\mat{Q}}
%\newcommand{\bV}{\mat{V}}
%\newcommand{\diag}{\textbf{diag}}

%\newcommand{\rme}{\mathrm{e}}
%\newcommand{\indm}{\ensuremath{\mathbf{1}_0}}
%\newcommand{\ef}{f}
%\newcommand{\vt}{\vartheta}

%\newcommand{\gd}{\delta}
%\newcommand{\gs}{\sigma}
%\newcommand{\gt}{\theta}
%\newcommand{\gb}{\beta}
%\newcommand{\ncal}{\mathcal{N}}
%\newcommand{\Gth}{\theta}

%\newcommand{\alex}[1]{\textcolor{red}{#1}}
%\newcommand{\ucal}{\mathcal{U}}
%\newcommand{\tv}{\vartheta}
%\newcommand{\zv}{z}


\maketitle

\vspace{-0.16in}

\begin{abstract} 
Allowing for some structure on a $p\times p$ matrix $\bQ$, we
derive semi-explicit solutions to constrained, quadratic
optimization problems of the form\footnote{Our results
generalize to additional linear constraints of the form $x^\top
a \ge b$ as well as a general lower bound constraints $x \ge
\ell$ for $\ell \in \bbR^p$ may be added but excluded here for
brevity. }
%form $x_i \le u_i$ are work in progress.}
\begin{equation} 
\begin{aligned}
 &\hspace{0.5in}\min_{x \in \bbR^p} x^\top \bQ x \\
  &\text{subject to: } \left\{
	\begin{array}{r}
	 \textstyle\sum\nolimits_k \hpt x_k =1 , \\
	\text{each }  x_k  \ge 0 . 
	\end{array} 
\right. \label{con}
\end{aligned}
\end{equation}

\hspace*{0.32in} These problems arise in a wide array of
scientific and engineering applications. The constraint 
$x_i \ge 0$ is often a crucial ingredient and in its presence, the
problem admits no (known) explicit solution in general.
A numerical solver must be used. 
This is disadvantageous for two  reasons we highlight.
\begin{enumerate}[label=\textit{(\arabic*)},
  itemindent=0.32in, itemsep=0.032in] {\it
  \item For $p$ large,
the computational efficiency may be unsatisfactory.
  \item Optimal points are not interpretable in the 
problem parameters. }
\end{enumerate}
\hspace*{0.32in} 
Analytic expressions address both of these disadvantages. 
We derive semi-closed-form solutions to $\req{con}$
under the assumption that $\bQ$ arises from a factor model.
That is, for a low-rank matrix $\bB \bV \bB^\top = 
\sum_{k=1}^q \pt \sigma^2_k \beta_k \beta_k^\top$
(i.e. rank $q \ll p$) where $\beta_k \in \bbR^p$ and
$\sigma_k \in \bbR$, we assume that
$\bQ - \bB \bV \bB^\top$ is a diagonal matrix 
and prove that the minimizer of $\req{con}$
is a function of $\bdel$ and
\begin{align}
 \hspace{0.32in} 
  \bB \hpt \theta = \theta_1 \beta_1 + \dots + \theta_q \beta_q
   \pt ;
 \hspace{0.32in} 
 \theta \in \bbR^q  \pt .
\end{align}
The vector $\theta = (\theta_1, \dots, \theta_q)$ is not known
explicitly, but may be recovered efficiently by solving a certain
fixed point problem (in dimension $q$).  The 
fixed point characterization of
$\theta$ also yields to interpretation of solutions 
in terms of each constituent factor $\beta_k$, an important
feature for many applications.


\hspace*{0.32in} Factor structure assumptions of the type 
above are pervasive in practice and give credence to 
statistical tools such as PCA. As an example we provide a
case study on the Markowitz portfolio problem with the no-short
sales constraint. Our results illustrate the 
computational advantages (item 1 above).
We leave questions of portfolio composition 
(item 2 above) to future work.

\end{abstract}


\section{Preliminaries} Consider a symmetric $p \times p$
matrix $\bQ$ given by 
\begin{align} \label{struct}
  \bQ = \bB \bV \bB^\top + \bdel
\end{align}
for a $p \times q$ matrix $\bB$ and diagonal matrices $\bV =
\diag\p{\sigma_1^2, \dots, \sigma_q^2}$ and $\bdel =
\textbf{diag}\p{\sv_1, \dots, \sv_p}$ with $\sv_i, \sigma_k^2
\in (0,\infty)$.  We regard $q$ as much smaller than $p$. 

To illustrate how $\bQ$, structured as in $\req{struct}$, arises
in applications, consider the covariance matrix of a linear
$q$-factor model over $p$ variables. That is, for a predictor $X =
(X_1, \dots, X_q)^\top$ we observe a $\bbR^p$-valued response
$Y$ defined by 
\begin{align}
  Y = \bB X + Z
\end{align} 
that is subject to errors $Z = (Z_1, \dots, Z_p)$. The
variables $X$ and $Z$ are random while $\bB$ is a constant
factor structure generating the correlation in the model.  If
the $\f{X_k,Z_i}$ are all uncorrelated (a standard assumption),
$\Var\p{Y}~=~\bQ$ for the right side equal to $\req{struct}$ 
whenever
every $\Var\p{X_k} = \sigma_k^2$ and every $\Var\p{Z_i} =
\delta_i^2$.  

\section{Main Results}
Our Main Theorem is:
\begin{theorem}\label{thm1}
Recall the minimization problem \req{con} .  Without loss of generality, we let $\bfB^\top \bdel^{-1} \rme \ge 0$.
Define  $\varphi \cst \bbR^q \to \bbR^q$ by
\begin{align}\label{phi}
\varphi (\vartheta) = \mat{V} \mat{B}^\top
\bdel^{-1} (\rme - \mat{B} \vartheta)_+
\end{align}
 Let $x$ be a solution of 
problem \req{con}. Then
\begin{align}
   x = \frac{w}{\prj{w}{\rme}}
  \hspace{0.16in}
  \text{where} 
  \hspace{0.16in} 
   w = \bdel^{-1} (\rme - \mat{B} \theta)_+
\end{align}
for $\theta \in \bbR^q$ the unique  fixed
point of $\varphi$ (i.e., $\varphi(\theta) = \theta$).
\end{theorem}
\begin{remark}
The condition such that $\bfB^\top \bdel^{-1} \rme \ge 0$ is without loss of generality, as we can flip the sign of any vector without altering the structure of the covariance matrix.
\end{remark}
\begin{remark}
$\varphi(\cdot)$ is not a contraction mapping; in fact, fixed-point iterations of $\varphi(\cdot)$ generally do not converge.
\end{remark}
Define $\psi \cst \bbR^q \to \bbR^q$ as 
$\psi (\vartheta) = \bfA^{-1}_\vartheta b_\vartheta$ where
\begin{equation}\label{linsys}
\begin{aligned}
  \bfA_\vartheta &= \bfV^{-1} + \bfB^\top \bdel^{-1}
  \textbf{diag}(\chi_\vartheta) \bfB \\
  b_\vartheta &= \bfB^\top \bdel^{-1} \chi_\theta \\
  \chi_\vartheta &= (\rme \ge \bfB \vartheta)   
  \in \{0,1\}^p
\end{aligned}
\end{equation}
\begin{lemma}
We have $\theta = \varphi(\theta)$ if and 
only if $\theta = \psi(\theta)$.
\end{lemma}
\begin{lemma}
The iterates $\vartheta^{k+1}
= \psi(\vartheta^k)$ converge to the fixed point 
of $\psi$ (i.e. $\theta = \psi(\theta)$)
provided that $\vartheta^0 = 0 \in \bbR^q$.
\end{lemma}
\begin{theorem}\label{thm:sol}
The iterates $\vartheta^{k+1}
= \psi(\vartheta^k)$ converge to the fixed point 
of $\psi$ (i.e. $\theta = \psi(\theta)$)
provided that $\vartheta^0 = 0 \in \bbR^q$.  This is also the fixed point $\varphi(\theta) = \theta$, which solves for the required $\theta$ in Theorem \ref{thm1}.
\end{theorem}
A similar expression for the Long/Short portfolio exists and omits the positivity correction, but this leads to a vastly different fixed point.  The portfolios correspond as follows:
\begin{table}
\centering
%\setlength{\tabcolsep}{0.02in}
\renewcommand{\arraystretch}{1.32}
  \begin{tabular}{ccc}
\toprule
  Quantity & 
  LS Min. Var. & LO Min. Var. \\
\midrule
 weights 
 & $x = \frac{w}{\prj{w}{\rme}}$
 & $x = \frac{w}{\prj{w}{\rme}}$ \\
 $w$ 
 & $w = \bdel^{-1} (\rme - \bfB \theta)$ 
 & $w = \bdel^{-1} (\rme - \bfB \theta)_+$ \\
 $\theta$ 
 & $\theta = \varphi(\theta)$
 & $\theta = \varphi(\theta)$ \\
 $\varphi(\vartheta)$ 
 & $\bfV \bfB^\top \bdel^{-1} (\rme - \bfB \vartheta)$ 
 & $\bfV \bfB^\top \bdel^{-1} (\rme - \bfB \vartheta)_+$ \\
 fixed point
 & $w = \Psi(w)$
 & $w = \Psi(w)$ \\
 $\Psi (v)$ 
 & $\bdel^{-1} (\rme - \bfB \bfV \bfB^\top v)$
 & $\bdel^{-1} (\rme - \bfB \bfV \bfB^\top v)_+$
 \\
\bottomrule
\end{tabular}
\end{table}

\section{Previous Work and the 1-Factor Case}
A similar solution has existed for a few years \cite{clarke2011}, but only considers a single factor case.  
\alex{With a single factor, we can generalize the original problem slightly
\begin{equation} 
\begin{aligned}
 &\hspace{0.5in}\min_{x \in \bbR^p} x^\top \bQ x \\
  &\text{subject to: } \left\{
  \begin{array}{r}
   \textstyle\sum\nolimits_k \hpt a_k x_k =1 , \\
  \text{each }  x_k  \ge 0 . 
  \end{array} 
\right. \label{1factorProb}
\end{aligned}
\end{equation}
where 
\begin{align*}
Q = \gb \gb^\T + \Gd
\end{align*}
and $\gb \in \bbR^p$ and $\Gd$ is a (strictly) positive-definite diagonal matrix.
}
The solution contained a similar formalism for the elements of the factor vector, with the un-normalized weights $w_i$ having the following form:
\begin{align}
w_i = \frac{1}{\gd_i^2}\paren{a - \theta \gb_i}_+
\end{align}
where $\theta$ is found as the fixed point of:
\begin{align}
 \psi_{\textsc{cdt}}(z) = \frac{1/\gs^2 + \sum_{\{i: \gb_i<z\}} \gb_i^2/\gd_i^2 }{ \sum_{\{i: \gb_i<z\}} \gb_i/\gd_i^2 }
\end{align}
This result is almost correct; a fixed point may not exist of $sum_i \gb_i<0$, and this is clearly a discontinuous map, so proving existence and uniqueness is still difficult.  Furthermore, for some values of $z$, $\psi_{CDT}$ may be $\pm \infty$.  Our map, $\psi$, is essentially the reciprocal of $\psi_{CDT}$, generalized to multiple dimensions with a slightly different condition on the summation.  Rather than checking if $\gb_i<z$, we check if $z \gb_i<1$.  Note that this alters the condition slightly, such that the signs of $\gb_i$ and $z$ affect which elements are being summed.  Also, our assumption that $\sum_i \frac{\gb_i}{\gd_i^2}>0$ guarantees the existence of a fixed point with our formulation.  Our fixed-point finding map is therefore:
\begin{align}\label{psi_1}
\psi(z) = \frac{\sum_{\{i:z\gb_i<1\}} \gb_i/\gd_i^2}{1/\gs^2+ \sum_{ \{i:z\gb_i<1\} } \gb_i^2/\gd_i^2}
\end{align}
This is simply the 1-dimensional version of Equation \req{linsys}.  This is still discontinuous, of course, but, as we have previously shown, the fixed points of this function correspond to the fixed points of 
\begin{align}\label{phi_1}
\phi(z) = \gs^2\paren{ \sum_{\{i:z\gb_i<1\}} 
\frac{\gb_i}{\gd_i^2}\paren{1 - z \gb_i}}.
\end{align}
This is, of course, also the 1-dimensional version of Equation \req{phi}.  This gives us the followinf set of lemmas and theorems:
\begin{lemma} 
The fixed-points of $\phi$ correspond with the fixed-points of $\psi$
\end{lemma}
\begin{proof}
The implication both ways is straightforward algebraic manipulation under the assumption that $\sum_i \frac{\gb_i}{\gd_i^2}>0$.
\end{proof}
\begin{lemma} The following are true about $\phi$:
\begin{enumerate}
\item $\phi(z)$ is continuous
\item $\phi(0)>0$
\item $\phi$ is differentiable a.e. and $\phi'(z)<0$ for all $z>0$
\item $\phi$ has a positive root
\end{enumerate}
\end{lemma}
\begin{proof}
These are all fairly straightforward to see in 1-dimension:
\begin{enumerate}
\item Continuity follows because the $\phi(\cdot)$ is piecewise-linear, and only changes slope when $z\gb_i = 1$, in which case the term being removed from the summation is identically $0$.
\item $\phi(0)>0$ follows from our assumption that $\sum_i \frac{\gb_i}{\gd_i^2}>0$; if $z\gb_i<1$ then 
\begin{align*}
\frac{\gb_i}{\gd_i^2}< (z\gb_i) \frac{\gb_i}{\gd_i^2}
\end{align*}
and at $z=0$, $\phi(z)$ must include all elements of $\gb$ as $0\gb_i <1$ .
\item Differentiability follows directly from piecewise linearity; the coefficient of $z$ is also negative, meaning that the derivative is piecewise constant and negative where it exists.  Furthermore, the derivative exists all points where $z\gb_i \neq 0$ for some $i$.
\item We have that $\phi(0)>0$ and $\phi$ has a piecewise constant negative derivative a.e., so $\phi$ must cross $0$ at some point of finite value.
\end{enumerate}
\end{proof}
\begin{theorem}
$\phi$ has a unique fixed point, and therefore so does $\psi$.
\end{theorem}
\begin{proof}
Applying the previous lemmas, we have that $\phi$ is continuous on a compact set, which guarantees existence of a fixed point via the Brouwer Fixed-Point Theorem.\\
Uniqueness follows from the fact that $\phi$ is monotonically decreasing on the entire compact set.
\end{proof}
%\alex{Add iterates description}

\iffalse
\begin{theorem}Starting from $0$, the iterates of $\psi(z)$ converge to a fixed point $\psi(z)=z$
\end{theorem}


\alex{We hypothesize $\psi(z)$ increases until the fixed point which corresponds with the maximum.  The function jumps at $\frac{1}{\gb_i}$ we need to understand the behavior at these jumps.  With one-factor, we can assume the $\gb_i$'s are sorted from largest to smallest, so that ($\frac{1}{\gb_1} \leq \frac{1}{\gb_2} \leq \frac{1}{\gb_3} \ldots$).  Our hypothesis also assumes that $\psi(0)>0$ which is without loss of generality.  First task: show that starting from $0$ and increasing in $z$, $\psi(z)$ either increases or hits the fixed-point.  Try to prove that $psi(z)$ does not decrease.\\
Use the following notation:
Recall:
\begin{align}
\psi(z) &= \frac{\sum_{\{i:z\gb_i<1\}} \gb_i/\gd_i^2}{1/\gs^2+ \sum_{ \{i:z\gb_i<1\} } \gb_i^2/\gd_i^2}
\end{align}
Note that for all $z\in [0,\frac{1}{\gb_1})$
\begin{align}
\psi(z) &= \frac{A+ \gb_1/\gd_1^2}{B+\gb_1^2/\gd_1^2}\\
&= \frac{\gd_1^2 A+ \gb_1}{\gd_1^2 B + \gb_1^2}\\
&=\psi(0)
\end{align}
where we define:
$A = \sum_{i=2}^p \frac{\gb_i}{\gd_i^2}$ and $B = \frac{1}{\gs^2}+\sum_{i=2}^p \frac{\gb_i^2}{\gd_i^2}$.
At the first jump and before the second jump where $z \in \left[\frac{1}{\gb_1},\frac{1}{\gb_2} \right)$, we have 
\begin{align}
\psi\left(\frac{1}{\gb_1}\right)=\frac{A}{B}
\end{align}
Either $\psi(z)$ has a fixed point on the interval $\left[0, \frac{1}{\gb_1}\right)$ or there are 2 possible cases : $\psi(z)$ jumps down, or it jumps up
\subsection*{Case 1}
Assume the function jumps down:
\begin{align}
\psi(0)= \frac{A+ \gb_1/\gd_1^2}{B+\gb_1^2/\gd_1^2} > \frac{A}{B}
\end{align}
the implication is that \sout{(\textbf{CHECK THIS})} It's true because $B,\gb_1,A >0$
\begin{figure}
\includegraphics[width=\linewidth]{graph.eps}
\end{figure}
\begin{align}
\frac{A}{B}<\frac{1}{\gb_1}
\end{align}
and therefore if $\frac{A}{B}< \frac{1}{\gb_1}$, $\psi(z)<z$ for $z \in \left[\frac{1}{\gb_1},\frac{1}{\gb_2}\right)$.  This means we jump below line of slope 1 (fixed point line)
We need to show 
\begin{enumerate}
\item $\psi(z)$ is decreasing for all $z>z_0$ such that $\psi(z_0)<z_0$ (below line $y=x$) OR:
\item show ``no future jump can exceed or match'' $\psi(z)=z$
\end{enumerate}
If either of these is shown, there is no fixed point for $\psi(z)$ which contradicts the earlier lemma/theorem which says that $\varphi(\cdot)$ has a fixed point and the fixed points of $\varphi(\cdot)$ corresponds with the fixed point $\phi(\cdot)$.  This contradiction therefore imples that before the fixed point, the function cannot jump down.\\
\proofsketch{Assume without loss of generality that $\gb_1, \gb_2>0$- this is allowed because if $z>0$, any negative $\gb_i$ will always be such that $z\gb_i<0$.  Further, assume w.l.o.g. that the fixed point falls in the interval $[0,\gb_1)$, that is, for some $z \in [0, \gb_1):$
\begin{align}
\psi(z) &= \frac{\sum_{\{i:z\gb_i<1\}} \gb_i/\gd_i^2}{1/\gs^2+ \sum_{ \{i:z\gb_i<1\} } \gb_i^2/\gd_i^2} = z < \frac{1}{\gb_1}
\end{align}
For simplicity, we will adapt the earlier notation under the assumption that $\gb_1>\gb_2>\ldots$ such that at at least some elements of $\gb_i>0$, including $\gb_1$ and $\gb_{2}$, and:
\begin{align}
A_i &= \sum_{k=2}^{n} \frac{\gb_{i}}{\gd_{i}^2}\\
B_i &= \sum_{k=2}^{n} \frac{\gb_{i}^2}{\gd_{i}^2},
\end{align}
For example:
\begin{align*}
A_1 = \sum_{k=2}^{n} \frac{\gb_{i}}{\gd_{i^2}} = \paren{\sum_{i=1}^n \frac{\gb_{i}}{\gd_{i}^2}} - \frac{\gb_{1}}{\gd_{1}^2}
\end{align*}
with a similar pattern for $B_1$.
In our new notation, we have that
\begin{align}
\frac{1}{\gb_1} > \frac{A_1}{B_1} = \frac{\gd_{2}^2 A_2 + \gb_{2}}{\gd_{2}^2 B_2 + \gb_{2}^2}
\end{align}
Note that because $\gb_1>0$ and $B_2>0$, we have
\begin{align}
 \paren{\gd_2^2 A_2 + \gb_2}\gb_1<\gd_2^2 B_2 + \gb_2^2 
\end{align}
rearranging terms, we have:
\begin{align}
\gd_2^2 \paren{A_2\gb_1 - B_2}<\gb_2 \paren{\gb_2 - \gb_1}
\end{align}
Because $\gb_1>\gb_2 >0$, we have that $\gb_2\paren{\gb_2-\gb_1}<0$, so
\begin{align}
A_2 \gb_1 - B_2 &<0\\
\frac{A_2}{B_2}< \frac{1}{\gb_1}
\end{align}
The condition that $\psi(z)$ is nondecreasing implies that 
\begin{align}
\frac{A_2}{B_2}-\frac{A_1}{B_1}  < 0
\end{align}
}
Assume not, that is, that
\begin{align}
A_2 (\gd_2^2 B_2 + \gb_2^2 )= A_2 B_1 > A_1 B_2 =(\gd_2^2 A_2 + \gb_2) B_2
\end{align}
which implies
\begin{align}
A_2 \gb_2^2 &> B_2 \gb_2\\
\implies A_2 \gb_2 - B_2 &> 0
\end{align}
However, $\gb_2 < \gb_1 \; \implies A_2 \gb_2 < A_2 \gb_1$. This contradicts that $A_2 \gb_1 -B_2 <0$.  It therefore follows that 
\begin{align}
\frac{A_2}{B_2} - \frac{A_1}{B_1} \leq 0
\end{align}
\subsection*{Case 2}
Assume the function jumps up:
\begin{align}
\psi(0)= \frac{A+ \gb_1/\gd_1^2}{B+\gb_1^2/\gd_1^2} < \frac{A}{B}
\end{align}
the implication is that (\textbf{CHECK THIS})
\begin{align}
\frac{1}{\gb_1}< \frac{A}{B}
\end{align}
either the next interval is higher or we hit the fixed point on this interval because an inductive extension of the Case 1 argument states that we cannot decrease if we have not hit the fixed point.
}
\newpage
\fi


Define:
\begin{align}
\psi(z) = \frac{\sum_{z \gb_i <1} \gb_i \gd_i^2}{1/\gs^2 + \sum_{z \gb_i<1} \paren{\gb_i/\gd_i}^2}
\end{align}
\begin{assumption}
We assume $\psi(0)=0$ and $\gb_1 > \gb_2 > \ldots > \gb_p$.
\end{assumption}
\begin{remark}
These assumptions are without loss of generality.
\end{remark}
\begin{remark}
We prove the following theorem for the interval $\left[0,\frac{1}{\gb_1}\right)$.  It generalizes inductively to positive intervals in $\bbR_+$.
\end{remark}
\begin{theorem}
Beginning from $z_0=0$, the fixed point iterations of $z_{k+1}=\psi(z_k)$ converge to a fixed point, where $\psi(z_k)$ is defined above.
\end{theorem}

\begin{proof}
First, we must define some notation.  Consider the following:
\begin{align*}
A_k &= \sum_{i={k+1}}^p \frac{\gb_i}{\gd_i^2}\\
B_k &= \frac{1}{\gs^2}+\sum_{i={k+1}}^p \frac{\gb_i^2}{\gd_i^2}
\end{align*}
Note that because $\gb_1>\gb_2>\ldots>\gb_p$, $\frac{1}{\gb_1}<\frac{1}{\gb_2}<\ldots$ for all of the values $\gb_i>0$.
Clearly, we have that $\psi(0)=\frac{A_0}{B_0}$.  Further, as long as $z<\frac{1}{\gb_1}$, we have $z \gb_1<1$, and so $\psi(z)$ is constant on the interval $[0,\frac{1}{\gb_1})$ (it will turn out, this is without loss of generality; $\psi(z)$ is constant between any interval $[\frac{1}{\gb_i},\frac{1}{\gb_{i+1}})$).  We will show that as $z$ increases from $0$, for $z \in [0,\frac{1}{\gb_1})$, the function $\psi(z)$ has two possibilities:
\begin{itemize}
\item $\psi(z)=z$ for some $z \in \left[0,\frac{1}{\gb_1}\right)$
\item If $\psi(z)\neq z$ for all $z \in \left[0,\frac{1}{\gb_1}\right)$, $\psi(z)$ jumps upwards for the interval $\left[\frac{1}{\gb_1},\frac{1}{\gb_2}\right)$; that is: the left limit $\psi\left(\frac{1}{\gb_1}^-\right)< \psi \left(\frac{1}{\gb_1}\right)$
\end{itemize}
Naively, it seems possible that $\psi\left(\frac{1}{\gb_1}^-\right)> \psi\left(\frac{1}{\gb_1}\right)$- that is, the function \textit{decreases} at $\frac{1}{\gb_1}$.  We will show this is not possible.\\
If it were possible (and $\psi(z)\neq z$ for $0<z<\frac{1}{\gb_1}$), we would have:
\begin{align}
\psi\paren{\frac{1}{\gb_1}^-} = \frac{A_0}{B_0}> \frac{A_1}{B_1} = \psi\paren{\frac{1}{\gb_1}}.
\end{align}
Assume, this were true, that is assume
\begin{align}
\frac{A_0}{B_0} > \frac{A_1}{B_1}
\end{align}
There are two cases:
\begin{enumerate}
\item $\frac{A_1}{B_1}>\frac{1}{\gb_1}$
\item $\frac{A_1}{B_1}< \frac{1}{\gb_1}$
\end{enumerate}
In Case 1, we have:
\begin{align}
\frac{A_0}{B_0}>\frac{A_1}{B_1}> \frac{1}{\gb_1}
\end{align}
where we note that:
\begin{align}
\frac{A_0}{B_0} &= \frac{\frac{\gb_1}{\gd_1^2}+\sum_{k=2}^p \frac{\gb_i}{\gd_i^2}}{\frac{1}{\gs^2}+ \frac{\gb_1^2}{\gd_1^2}+ \sum_{k=2}^p \frac{\gb_i^2}{\gd_i^2}} = \frac{\gd_1^2 A_1 + \gb_1^2}{\gd_1^2 B_1 + \gb_1^2}\\
&= \frac{A_1 + \frac{\gb_1}{\gd_1^2}}{B_1 + \frac{\gb_1^2}{\gd_1^2}}
\end{align}
It therefore follows that:
\begin{align}
\frac{A_0}{B_0} &= \frac{A_1 + \frac{\gb_1}{\gd_1^2}}{B_1 + \frac{\gb_1^2}{\gd_1^2}}> \frac{A_1}{B_1}\\
\implies \frac{\gb_1}{\gd_1^2}B_1 &> \frac{\gb_1^2}{\gd_1^2} A_1
\end{align}
It therefore follows that 
\begin{align}
\frac{A_1}{B_1}< \frac{1}{\gb_1}
\end{align}
which contradicts our assumption that $\frac{A_1}{B_1}> \frac{1}{\gb_1}$.  We must now check Case 2.
In Case 2, we have:
\begin{align}
\frac{A_1}{B_1} < \frac{1}{\gb_1}
\end{align}
By similar computations to those with $A_0$ and $B_0$, we have:
\begin{align}
\frac{A_1}{B_1} = \frac{\gd_2^2 A_2 + \gb_2}{\gd_2^2 B_2 + \gb_2^2} < \frac{1}{\gb_1}
\end{align}
\begin{assumption}
we proceed under the assumption that $\gb_1>\gb_2>0$ and $A_1, A_2>0$.  
\end{assumption}
If $\gb_2\leq 0$, our ordering assumption means that $\gb_3, \ldots <0$, and so $A_2<0$ and so $\frac{A_1}{B_1}<0$.  We therefore have that $\psi(z)<0$ for all $z>\frac{1}{\gb_1}$, and so we never have a fixed point.  Therefore, under the assumption that $\gb_2>0$, our previous assertion implies that
\begin{align}
\gb_1 \paren{\gd_2^2 A_2 + \gb_2} < \gd_2^2 B_2 + \gb_2^2
\end{align}
Rearranging terms, we have
\begin{align}
\gd_2^2 \paren{A_2 \gb_1 - B_2}<\gb_2\paren{\gb_2-\gb_1}
\end{align}
However, $\gb_2<\gb_1$, so this implies that
\begin{align}
A_2 \gb_1- B_2 &<0\\
\implies \frac{A_2}{B_2}< \frac{1}{\gb_1}
\end{align}
Now, assume that
\begin{align}
\frac{A_2}{B_2}> \frac{A_1}{B_1}
\end{align}
we are only assuming this, positivity of $\gb_1, \gb_2$, and that $\frac{A_1}{B_1}< \frac{1}{\gb_1}$ (although it follows from our previous result that $\frac{A_1}{B_1}< \frac{1}{\gb_1}$).  We have
\begin{align}
\frac{A_2}{B_2}>\frac{\gd_2^2 A_2 + \gb_2}{\gd_2^2 B_2 + \gb_2^2}
\end{align}
and it follows that
\begin{align}
A_2 \paren{\gd_2^2 B_2 + \gb_2^2} &> B_2\paren{\gd_2^2 A_2 + \gb_2} \\
\implies A_2 \gb_2^2  &> B_2  \gb_2
\end{align}
and so it follows that $\paren{A_2 \gb_2 - B_2}>0$.  However, from our earlier assumption, that $\gb_1>\gb_2$, note that $A_2 \gb_1 > A_2 \gb_2$.  Therefore, this contradicts that $A_2 \gb_1-B_2<0$.  Therefore Case 2 cannot happen either.\\
It therefore follows that $\psi(z)$ either has a fixed point in $[0,\frac{1}{\gb_1})$ or increases when $z> \frac{1}{\gb_1}$.
\end{proof}
\newpage
\section{Sensitivity Analysis of Unnormalized Weights at Fixed Point}
We wish to study the sensitivity of both the normalized weights $x_i$ and the un-normalized weights $w_i$ to the parameters $\gs^2,\gb_j$ and $\gd_i^2$ at the fixed point, i.e. where 
\begin{align}
\theta = \psi(\theta)= \gf(\theta).
\end{align}
Note that $\psi(\cdot)$, $\gf(\cdot)$ are defined above.  Note that at the fixed point, 
\begin{align}
\theta=\gf(\theta)
\end{align}
and 
\begin{align}
\diff{\alpha} \theta = \diff{\alpha} \phi(\theta)
\end{align}
where $\alpha \in \{\gs^2, \gb_j, \gd_i^2 \}$.  We will take advantage of this fact for each of our computations.\par
To be more precise, define:
\begin{align}
\theta &= \theta(\ga)\\
\diff{\ga} \theta(\ga) &= \lim_{h \to 0} \frac{\theta(\ga+h)-\theta(\ga)}{h}
\end{align}
First we consider the derivative of $x_i$ and $w_i$ with respect to $\gs^2$.
\newpage
\section{Algorithm}  Algorithm \ref{alg} computes the
solution in Theorem \ref{thm:sol}.
It implements the fixed point iterations
$\theta^{n+1} = \psi\p{\theta^n}$ to compute 
$\psi (\gt) = \gt$.

\newcommand{\told}{\theta_{\text{old}}}
\newcommand{\tnew}{\theta_{\text{new}}}

\begin{algorithm}[hpt!]
\vspace{0.16in}
\begin{enumerate}[label=\textbf{\arabic*.},
  itemindent=0.32in, itemsep=0.032in]
\item Initialize $\told \leftarrow (0, \dots, 0) \in \bbR^q$ 
and tolerance $\epsilon > 0$.
\item Assemble $\mat{A}_{\told}$ and $b_{\told}$ 
from 
$(\bB, \bV, \bdel)$ and $\told$ as in $\req{linsys}$. \label{loop}
\item Compute $\tnew$ by solving 
$\mat{A}_{\told} \tnew = b_{\told}$. 
\item If $| \tnew - \told | >\epsilon$, 
update $\told \leftarrow \tnew$ and go to Step \ref{loop}.
\item Compute $w=\bdel^{-1}(\rme-\mat{B} \tnew)_+$
and {\bf return} $\frac{w}{\rme^\top w}$.
\end{enumerate}
\caption{(FFP). Given $(\mat{B}$, $\mat{V}$, $\bdel)$, computes
the optimizer of problem \req{con}. }
\label{alg}
\end{algorithm}
We refer to Algorithm \ref{alg} as the \emph{factor fixed point}
(FFP) method. The considerable gain in efficiency of the FFP
algorithm over traditional quadratic programming solvers is due
to the fact that it requires only solutions to a $q \times q$
linear system. Traditional solvers require computations on
the full problem size $p \times p$. 


 Under our assumptions on $\bQ$, the FFP algorithm
has a running time of $O(pq + q^2)$. The time consuming
operations (those of order $pq$) involve assembling the linear
system in $\req{linsys}$. This complexity
may potentially be reduced by solving $\req{linsys}$
explicitly via Woodbury type identities and precomputing some of
the variables. We do not pursue this here. We also leave open
the question of convergence of the FFP iterations. We treat them
as constant (relative to $p$).



\newpage
\appendix


\section{Python Pseudocode} \hspace{1mm} \\
\verb|Python| implementation of the function $\psi$.
\begin{verbatim}
def psi(theta):

    chi = Dinv * (ones >= B @ theta)
    b = B.T @ chi
    A = Vinv + (B.T @ np.diag(chi)) @ B
    
    return solve(A,b)


\end{verbatim}

\verb|Python| implementation of the fixed point iterations.

\begin{verbatim}
def compute_fixed_point (t):

    s = t + 2 * tol

    while (norm(s - t) > tol * norm(s)):
        t = s
        s = psi(t)

    return(s)
\end{verbatim}

\verb|Python| computation of the long-only portfolio weights.

\begin{verbatim}
def lo_weights ():

    theta = psi (np.zeros(q))
    w = Dinv @ maximum(ones - B @ theta, 0) 
    
    return w / sum(w)

\end{verbatim}

\bibliographystyle{agsm}
\bibliography{biblio}

\end{document}






